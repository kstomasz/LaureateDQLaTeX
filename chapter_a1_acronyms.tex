%=====================================================
% Acronyms
%=====================================================

%-----------------------------------------------------
% Samples
%-----------------------------------------------------

% Usage:
% \gls{AD} is pretty interesting. If we do reference a glossary entry,
% like, \gls{glos:AD}, that one of course has to be defined over there.

% An acronym with a glossary entry not hyperlinked
% \newacronym{AD}{AD}{Active Directory\protect\glsadd{glos:AD}}

% Geek Version, with hyperlink to glossary
% \newglossaryentry{AD}{
%   type=\acronymtype,
%   name=AD,
%   first=Active Directory (AD),
%   firstplural={Active Directories (AD's)},
%   description=\glslink{glos:AD}{Active Directory}
% }

%----------------------------------------------------- 
% Content
%-----------------------------------------------------

%<content>%

\newglossaryentry{sox}{
   type=\acronymtype,
   name=SOX,
   first=Sarbanes--Oxley Act (SOX),
   description=\glslink{glos:sox}{Sarbanes--Oxley Act}
 }

\newglossaryentry{sme}{
   type=\acronymtype,
   name=SME,
   first=Small and Medium Enterprise (SME),
   firstplural=Small and Medium Enterprises (SMEs),
   description=\glslink{glos:sme}{Small and Medium Enterprise}
 }

\newglossaryentry{ifrs}{
   type=\acronymtype,
   name=IFRS,
   first=International Financial Reporting Standard (IFRS),
   firstplural=International Financial Reporting Standards (IFRS),
   description=\glslink{glos:ifrs}{International Financial Reporting Standard}
 }
 
 \newglossaryentry{gaap}{
   type=\acronymtype,
   name=GAAP,
   first=Generally accepted accounting principles (GAAP),
   firstplural=Generally accepted accounting principles (GAAP),
   description=\glslink{glos:gaap}{Generally accepted accounting principles}
 }
 
 \newglossaryentry{xbrl}{
   type=\acronymtype,
   name=XBRL,
   first=eXtensible Business Reporting Language (XBRL),
   firstplural=eXtensible Business Reporting Language (XBRL),
   description=\glslink{glos:xbrl}{eXtensible Business Reporting Language}
 }
 
 
 
 
 

%</content>%
  
  























