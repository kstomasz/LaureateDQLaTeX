%=====================================================
% Glossary
%=====================================================

%-----------------------------------------------------
% Samples
%-----------------------------------------------------

% Usage:
% \gls{glos:AD} is pretty interesting. If we have a cross reference from
% the acronyms, we can also directly go to that using \gls{AD}; this
% requires then that over there, we have something like
%  description=\glslink{glos:AD}{Active Directory}

% \newglossaryentry{glos:AD}{
% name=Active Directory,
% description={Active Directory is the directory service for 
% Windows based networks, that allows central organization and 
% administration of any network resource.
% It allows a single-sign-on concept independent from network 
% topologies or network protocols. As a prerequisite you need 
% a Windows Server acting as Domain Controller. This computer 
% stores all necessary data, e.\,g.~usernames and corresponding 
% passwords.}
% }


%-----------------------------------------------------
% Content
%-----------------------------------------------------

%<content>%
% \renewcommand{\theHequation}{\theHsection.\equationgrouping\arabic{equation}}

%------------------------------------------------------------------------------------------------------------------------

\newglossaryentry{glos:sox}{
 name=Sarbanes--Oxley Act,
 description={``An act to protect investors by improving the accuracy and reliability of corporate disclosures July 30, 2002 made pursuant to the securities laws, and for other purposes''  \citep[1]{United-States-Government:2002vn}, the Sarbanes--Oxley Act brought several new regluations such as that the CEO and CFO are required to certify the accuracy of corporate financial reports which have published timely along with reports on the effectiveness of the company's internal control structures and financial reporting procedures.}}

%------------------------------------------------------------------------------------------------------------------------

\newglossaryentry{glos:sme}{
name=Small and Medium Enterprise,
description={Company whose personnel number falls under
certain thresholds, typically in the range of maximum 100--250
people, depending on the jurisdiction.}}

%------------------------------------------------------------------------------------------------------------------------

\def\descstakeholder{According to \citeauthor{Donaldson:1995uq}, anyone
who has a ``stake, (i.e., potential benefit)''  (\citeyear[86]{Donaldson:1995uq}), in the 
company, i.e., they consider
the Stanford Research Institute's definition of
stakeholders as ``those groups without whose support the organization would cease to exist''
(\citeyear[72]{Donaldson:1995uq}) as too narrow,  and while they hold that ``it is
essential to draw a clear line between influencers and stakeholders,''
(\citeyear[loc. cit.]{Donaldson:1995uq}) they see managers not only as
``referees'' between the two stakeholder groups of investors and employees, but emphasize
``the fact that managers of a firm are one of its most important and powerful
constituencies.'' (\citeyear[loc.
cit.]{Donaldson:1995uq}) The analysis that
\citeauthor{Donaldson:1995uq} take is oriented towards a normative argument 
which works out the responsibility of managers in that they ``\emph{should}
acknowledge diverse stakeholder interests and  \emph{should} attempt to respond
to them within a mutually supportive framework because that's a moral
requirement for the legitimacy of the management function.'' 
(\citeyear[p. 87, emphasis theirs]{Donaldson:1995uq}).}
\newglossaryentry{glos:stakeholder}{
name=stakeholder,
firstplural=stakeholders,
description={\descstakeholder}}

\newglossaryentry{glos:agencytheory}{
name=Agency Theory,
description={According to \citeauthor{Kolb:2008fk},
``the theory of agency seeks to explain why and how service and control can succeed or fail \dots One actor, the agent, is modeled as acting for another, the principal. \dots The actor's problems focus on serving
[or not] the principal; the principal's problems generally entail dilemmas of how to assure that the agent will do what the principal wants him or her to do.'' (\citeyear[42]{Kolb:2008fk})}}

%------------------------------------------------------------------------------------------------------------------------

\def\glosdescifrs{International Financial Reporting Standards (IFRS) are designed as a common global language for business affairs so that company accounts are understandable and comparable across international boundaries. They are a consequence of growing international shareholding and trade and are particularly important for companies that have dealings in several countries. They are progressively replacing the many different national accounting standards.The rules to be followed by accountants to maintain books of accounts which is comparable, understandable, reliable and relevant as per the users internal or external.}
\newglossaryentry{glos:ifrs}{
  name=International Financial Reporting Standard (IFRS),
  firstplural=International Financial Reporting Standards (IFRS),
  plural=International Financial Reporting Standards,
  description={\glosdescifrs}
}

\def\glosdescgaap{Generally accepted accounting principles (GAAP) refer to the standard framework of guidelines for financial accounting used in any given jurisdiction; generally known as accounting standards or standard accounting practice. These include the standards, conventions, and rules that accountants follow in recording and summarizing and in the preparation of financial statements.}
\newglossaryentry{glos:gaap}{
  name=Generally accepted accounting principles (GAAP),
  firstplural=Generally accepted accounting principles (GAAP),
  plural=Generally accepted accounting principles (GAAP),
  description={\glosdescgaap}
}

\def\glosdescxbrl{eXtensible Business Reporting Language (XBRL) is a standards-based way to communicate and exchange business information between business systems. These communications are defined by metadata set out in taxonomies, which capture the definition of individual reporting concepts as well as the relationships between concepts and other semantic meaning.}
\newglossaryentry{glos:xbrl}{
  name=Extensible Business Reporting Languages (XBRL),
  firstplural=eXtensible Business Reporting Languages (XBRL),
  plural=eXtensible Business Reporting Languages (XBRL),
  description={\glosdescxbrl}
}

%------------------------------------------------------------------------------------------------------------------------

\def\glosdescaccrualsaccounting{``Accruals accounting is preparing the income statement and statement of financial position following the \gls{glos:accrualsconvention}, which says that profit = revenue - expenses (not cash receipts - cash payments).'' \citep[107]{Atrill:2011qy}}
\newglossaryentry{glos:accrualsaccounting}{
  name=accruals accounting,
  firstplural=accruals accountings,
  plural=accruals accountings,
  description={\glosdescaccrualsaccounting}
}


\def\glosdescaccrualsconvention{``The convention of accounting that asserts that profit is the excess of revenue over expenses, 
not the excess of cash receipts over cash payments.'' \citep[632]{Atrill:2010ys}}
\newglossaryentry{glos:accrualsconvention}{
  name=accruals convention,
  firstplural=accruals conventions,
  plural=accruals conventions,
  description={\glosdescaccrualsconvention}
}

%------------------------------------------------------------------------------------------------------------------------

\def\glosdesccurrentratio{According to \citeauthor{Atrill:2010ys}, ``the
current ratio compares the `liquid' asssets \dots\, of the business with the
current liabilities. The ratio is calculated as:''

\begin{equation}
  \text{Current Ratio} = \frac{\text{Current Assets}}{\text{Current Liabilities}}
\end{equation}

It should ideally have a value around 2  (\citeyear[160]{Atrill:2010ys}).
}
\newglossaryentry{glos:currentratio}{
  name=current ratio,
  firstplural=current ratios,
  plural=current ratios,
  description={\glosdesccurrentratio}
}


\def\glosdescacidtestratio{
According to \citeauthor{Atrill:2006ly}, ``the acid test ratio is very similar to
the \gls{glos:currentratio}, but it represents a more stringent test of liquidity'' excluding
inventories that ``cannot be converted into cash quickly.'' It is calculated as:


\begin{subequations}
\begin{align}
  \text{Acid Test Ratio} &= \frac{\text{Current Assets} - \text{Inventories}}{\text{Current Liabilities}}
  \intertext{and with}
  \text{Current Ratio} &= \frac{\text{Current Assets}}{\text{Current Liabilities}}
  \intertext{it follows}
  \text{Acid Test Ratio} &= \text{Current Ratio}- \frac{\text{Inventories}}{\text{Current Liabilities}} \label{eqn:cr_acid}
\end{align}
\end{subequations}

It should normally have a value above 1 (\citeyear[188]{Atrill:2006ly}). Equation \eqref{eqn:cr_acid}
shows the dependence between acid test ratio and current ratio; in particular, the second part, the relation
between inventories and current liabilities, can be understood as follows: a low value (or a high value of
the ratio of current liabilities per inventories) indicates an over-reliance on unsold goods to finance operations. 
It is also known as quick ratio or liquid ratio.
}
\newglossaryentry{glos:acidtestratio}{
  name=acid test ratio,
  firstplural=acid test ratios,
  plural=acid test ratios,
  description={\glosdescacidtestratio}
}

\def\glosdescliquidity{Liquidity is the ability to pay short-term obligations
when they fall due. ``It is vital to the survival of a business that there are
sufficient liquid resources available to meet maturing obligations (that is,
amounts owing that must be paid in the near future).''
\citep[141]{Atrill:2010ys} It is measured as
\glslink[hyper=true]{glos:currentratio}{\gls{glos:currentratio}} or
\gls[hyper=true]{glos:acidtestratio}.
}
\newglossaryentry{glos:liquidity}{
  name=liquidity,
  firstplural=liquidities,
  plural=liquidities,
  description={\glosdescliquidity}
}

%------------------------------------------------------------------------------------------------------------------------

\def\glosdescprofitability{The quality of affording gain or profit. ``Businesses
generally exist with the primary purpose of creating wealth for their owners.
Profitability ratios provide an insight to the degree of success in achieving
this purpose. They express the profit made (or figures bearing on profit, such
as sales revenue or particular expenses, like labour cost) in relation to other
key figures in the financial statements or to some business resource.'' \citep[141]{Atrill:2010ys}
It is measured as \glslink[hyper=true]{glos:rosf}{\gls{glos:rosf}},  \glslink[hyper=true]{glos:roce}{\gls{glos:roce}},
 \glslink[hyper=true]{glos:netprofitmargin}{\gls{glos:netprofitmargin}} and
  \glslink[hyper=true]{glos:grossprofitmargin}{\gls{glos:grossprofitmargin}}. 
}
\newglossaryentry{glos:profitability}{
  name=profitability,
  firstplural=profitabilities,
  plural=profitabilities,
  description={\glosdescprofitability}
}

\def\glosdescrosf{``The return on ordinary shareholder's funds compares the amount
of profit for the period available to the owners, with the owners' average stake in
the business during that same period'' \citep[174]{Atrill:2006ly}: 

\begin{equation}
  \text{ROSF} = \frac{\text{Net profit after taxation and preference dividend (if any)}}{\text{Ordinary share capital plus reserves}} \times 100~\%
\end{equation}

The value should be as high as possible as long as ``it is not achieved at the
expense of future returns by, for example, taking on more risky activities.''
(\citeyear[175]{Atrill:2006ly}) 
}

\newglossaryentry{glos:rosf}{
  name=Return on Ordinary Shareholder's Funds (ROSF),
  firstplural=Returns on Ordinary Shareholder's Funds (ROSF),
  plural=Returns on Ordinary Shareholder's Funds (ROSF),
  description={\glosdescrosf}
}

\def\glosdescroce{``The return on capital employed \dots\, expresses the relationship
between the net profit generated during a period and the average long-term capital
invested in the business during that period'' \citep[175]{Atrill:2006ly}:

\begin{equation}
  \text{ROCE} = \frac{\text{Net profit before interest and taxation}}{\text{Share Capital} + \text{Long term loans}} \times 100~\%
\end{equation}

``ROCE is considered to be the main measure of profitability. It compares inputs (capital
invested) with outputs (profit).'' (\citeyear[176]{Atrill:2006ly})
}

\newglossaryentry{glos:roce}{
  name=Return on Capital Employed (ROCE),
  firstplural=Returns on Capital Employed (ROCE),
  plural=Returns on Capital Employed (ROCE),
  description={\glosdescroce}
}


\def\glosdescnetprofitmargin{
  ``The net profit margin ratio relates the net profit for the period to the sales revenue
  during that period:''
  
  \begin{equation}
    \text{Net profit margin} = \frac{\text{Net profit before interest and taxation}}{\text{Sales revenue}} \times 100~\%
  \end{equation}

  ``This is often regarded as the most appropriate measure of operational performance \dots\, because
  differences arising from the way in which the business is financed will not influence the measure. \dots\,
  The ratio can vary considerably between types of business.'' \citep[177]{Atrill:2006ly}
}
\newglossaryentry{glos:netprofitmargin}{
  name=Net Profit Margin,
  firstplural=Net Profit Margins,
  plural=Net Profit Margins,
  description={\glosdescnetprofitmargin}
}


\def\glosdescnetprofit{
The net profit, also known as the ``Bottom Line'', the net income or the net earnings of a company, measures
the profitability of a company after accounting for all costs:

\begin{subequations}
\begin{align}
\text{Net Profit} &= \underbrace{\underbrace{\text{Sales Revenue} - \text{COGS}}_{\text{Gross Profit}} - \text{SG\&A} - \text{R\&D}}_{\text{Operating Profit}} - \text{ITD} \label{eqn:netprofit}
\end{align}
\end{subequations}

In equation \eqref{eqn:netprofit}, the \emph{ITD} refers to interest, taxes and depreciation expenses.
}

\newglossaryentry{glos:netprofit}{
  name=Net Profit,
  firstplural=Net Profits,
  plural=Net Profits,
  description={\glosdescnetprofit}
}



% \newcommand*{\aboxed}[2]{%
%   \rlap{\boxed{#1#2}}%
%   \phantom{\hskip\fboxrule\hskip\fboxsep #1}&\phantom{#2}%
% }


%=================================================================================
% Gross Profit Margin
%=================================================================================
\def\glosdescgrossprofitmargin{``The gross profit margin ratio relates
the gross profit of the business to the sales revenue generated over the same period
\dots\, The ratio is therefore a measure of profitability in buying (or producing)
and selling goods before any other expenses:'' \citep[178]{Atrill:2006ly}

\begin{subequations}
\begin{align}
  \text{Gross Profit Margin} &= \frac{\text{Gross Profit}}{\text{Sales Revenue}} 
  \intertext{and with}
  \text{Gross Profit} &= \text{Sales Revenue} - \text{\glslink{cogs}{COGS}}
  \intertext{it follows}
\tikzmarkin{a}(5.4,-0.5)(-1.0,0.75)\text{Gross Profit Margin} &=  1 - \frac{\text{\gls{cogs}}}{\text{Sales Revenue}} \label{eqn:grossprofitmargin}\tikzmarkend{a}
\end{align}
\end{subequations}


Equation \eqref{eqn:grossprofitmargin} features the relation of cost of goods sold to revenue,
also known as \glslink{glos:costofgoodstosales}{cost of goods sold ratio}. The gross profit
margin is normally expressed as a percentage.
}
\newglossaryentry{glos:grossprofitmargin}{
  name=Gross profit margin,
  firstplural=Gross profit margins,
  plural=Gross profit margins,
  description={\glosdescgrossprofitmargin}
}





%=================================================================================
% COGS
%=================================================================================

\def\glosdesccogs{Cost of Goods Sold (COGS) or cost of sales refers to the inventory costs 
of the goods a business has sold during a given period of time. These costs include all costs
of purchase, conversion and other costs including relocation of the goods into their current
inventories and stocking condition: These costs may include material, labor, and overhead
specifically allocated to the goods.}
\newglossaryentry{glos:cogs}{
  name=Cost of Goods Sold,
  description={\glosdesccogs}
}

%=================================================================================
% COGS to Sales Ratio
%=================================================================================

\def\glosdesccostofgoodstosales{
  The COGS to Sales Ratio or Cost of Goods to Sales Ratio shows the percentage of sales revenue that is used
  to pay for expenses which vary in direct relation to sales activities:
  
  \begin{equation}
    \text{COGS Ratio} = \frac{\text{COGS}}{\text{Sales revenue}} \label{eqn:cogstosales}
  \end{equation}
  
  A stable ratio over time can indicate that
  the company is effective at controlling its \glslink{glos:grossprofitmargin}{gross profit margin}.
}
\newglossaryentry{glos:costofgoodstosales}{
  name=Cost of Goods to Sales Ratio,
  description={\glosdesccostofgoodstosales}
}

%=================================================================================
% Markup Ratio
%=================================================================================

\def\glosdescmarkupratio{
The markup ratio, in relation to one given product, is the difference between the selling price and the
buying price of a product, expressed as a percentage of the buying price of that product. If calculated
over all selling activities a company performed within a given period of time, the markup ratio for
all goods sold can be calculated as follows:

\begin{subequations}
\begin{alignat}{4}
   &  \text{Markup Ratio} &&=  \frac{\text{Sales Revenue} - \text{COGS}}{\text{COGS}} \\[0.2cm]
  \intertext{hence}\\[-\baselineskip]
   & \tikzmarkin{b}(3.2,-0.5)(-3.2,0.75) \text{Markup Ratio}  &&= \frac{\text{Sales Revenue}}{\text{COGS}} - 1  \label{eqn:markup}\tikzmarkend{b}\\[0.2cm]
\intertext{Equation \eqref{eqn:markup} is particularly interesting as it mirrors equation \eqref{eqn:grossprofitmargin}.
In particular, it follows that both Markup Ratio and Profit Margin are dependent, and it is sufficient to know
one of them: Let's solve equation \eqref{eqn:grossprofitmargin} and use some abbreviations:}
          & m &&=  \text{Gross Profit Margin}\\
          &  u  &&= \text{Markup Ratio}\\
  \eqref{eqn:grossprofitmargin} \RA & 1 - m &&= \frac{\text{COGS}}{\text{Sales Revenue}}\\[0.2cm]
  \LRA & \frac{1}{1 - m} &&=  \frac{\text{Sales Revenue}}{\text{COGS}}\\[0.2cm]
  \intertext{If we now substitute this into equation \eqref{eqn:markup}, we get:}
  \eqref{eqn:markup} \RA & u &&=  \frac{1}{1 - m}  - 1\\[0.2cm]
  \intertext{Doing some basic math, we get:}
  \LRA & 1 + u &&= \frac{1}{1 - m}   \\[0.2cm]
  \LRA & \left(1 + u \right) \left(1 - m \right) &&= 1\\[0.2cm]
  \LRA & u - m  &&= m u \\[0.2cm]
  \LRA & 1 - \frac{m}{u} &&= m \\[0.2cm]
  \LRA & u = \frac{m}{1 - m}\\[0.2cm]
  \intertext{or, unabbreviated,}
   \LRA & \tikzmarkin{c}(2.1,-0.5)(-3.2,0.75) \text{Markup Ratio} &&= \frac{1 - \text{Gross Profit Margin}}{\text{Gross Profit Margin}}\label{eqn:markup2}\tikzmarkend{c}\\[0.2cm]
  \intertext{conversely,}
  \LRA & \tikzmarkin{d}(3.17,-0.5)(-3.2,0.75) \text{Gross Profit Margin} &&= \frac{\text{Markup Ratio}}{\text{Markup Ratio} + 1}\label{eqn:pm2}\tikzmarkend{d}\\[0.2cm]
 \end{alignat}
\end{subequations}

Equations \eqref{eqn:markup2} and \eqref{eqn:pm2} can be used to convert markup ratio into 
gross profit margin and vice versa. They show pretty clearly the relation between both ratios.
}
\newglossaryentry{glos:markupratio}{
  name=Markup Ratio,
  description={\glosdescmarkupratio}
}

%=================================================================================
% Operating Profit Margin
%=================================================================================


\def\glosdescoperatingprofitmargin{
The operating profit margin is defined as
\begin{subequations}
\begin{align}
\text{Operating Profit Margin} &= \frac{\text{Operating Profit}}{\text{Sales Revenue}}
\intertext{where}
\text{Operating Profit} &= \text{Gross Profit} - \text{SG\&A} - \text{R\&D}
\intertext{hence}
\begin{split}\text{Operating Profit Margin} &= \text{Gross Profit Margin}\\ & - \frac{\text{SG\&A}}{\text{Sales Revenue}} - \frac{\text{R\&D}}{\text{Sales Revenue}} \label{eqn:opm}\end{split}
\end{align}
\end{subequations}

Looking at equation \eqref{eqn:opm}, the operating margin, also called EBITDA margin, can be identified
as the gross profit margin less the \gls{sga} to sales ratio---the cost of doing business per revenue---and the
R\&D to sales ratio---also referred to as ``research intensity" (see \citeauthor{Hundley:1996kx} with regards
to the relationship between research intensity, profitability and liquidity).
}
\newglossaryentry{glos:operatingprofitmargin}{
  name=Operating Profit Margin,
  firstplural=Operating Profit Margins,
  plural=Operating Profit Margins,
  description={\glosdescoperatingprofitmargin}
}

%=================================================================================
% SG&A
%=================================================================================

\def\glosdescsga{
The Selling, General and Administrative Expenses (SG\&A) are major non-productive costs presented in
the income statement. They consist of costs for operating the business, split into:

\begin{itemize}
  \item \emph{Selling:} Cost of Sales, which includes salaries, marketing, rent, and all expenses
  as well as taxes related to producing and selling the products.
  \item \emph{General:} General operating expenses as well as taxes in relation to the operation
  of the company.
  \item \emph{Administration:} Executive salaries, support and associated taxes in relation to the
  administration of the company.   
\end{itemize}
}
\newglossaryentry{glos:sga}{
  name={Selling, General and Administrative Expenses (SG\&A)},
  firstplural={Selling, General and Administrative Expenses (SG\&A)},
  plural={Selling, General and Administrative Expenses},
  description={\glosdescsga}
}


%=================================================================================
% Working Capital
%=================================================================================

\def\glosdescworkingcapital{According to \citeauthor{Losbichler:2012zr},
``working capital is defined as the sum of inventories and customer
receivables, less supplier liabilities, and is often measured by the cash-to-cash (C2C) cycle time.
}
\newglossaryentry{glos:workingcapital}{
  name=Working Capital,
  description={\glosdescworkingcapital}
}


\def\glosdescroa{
The Return on Assets (ROA) indicates how profitable a company is related to its total assets,
i.e. how efficient its management is at using the assets to generate earnings:

\begin{equation}
\text{Return on Asset} = \frac{\text{\gls{glos:netprofit}}}{\text{Total Assets}}
\end{equation}

The Return on Assets is often referred to as Return on Investment (ROI).
}
\newglossaryentry{glos:roa}{
  name=Return on Assets (ROA),
  description={\glosdescroa}
}




%</content>%























