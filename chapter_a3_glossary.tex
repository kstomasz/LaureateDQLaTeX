%=====================================================
% Glossary
%=====================================================

%-----------------------------------------------------
% Samples
%-----------------------------------------------------

% Usage:
% \gls{glos:AD} is pretty interesting. If we have a cross reference from
% the acronyms, we can also directly go to that using \gls{AD}; this
% requires then that over there, we have something like
%  description=\glslink{glos:AD}{Active Directory}

% \newglossaryentry{glos:AD}{
% name=Active Directory,
% description={Active Directory is the directory service for 
% Windows based networks, that allows central organization and 
% administration of any network resource.
% It allows a single-sign-on concept independent from network 
% topologies or network protocols. As a prerequisite you need 
% a Windows Server acting as Domain Controller. This computer 
% stores all necessary data, e.\,g.~usernames and corresponding 
% passwords.}
% }


%-----------------------------------------------------
% Content
%-----------------------------------------------------

%<content>%


\newglossaryentry{glos:sox}{
 name=Sarbanes--Oxley Act,
 description={``An act to protect investors by improving the accuracy and reliability of corporate disclosures July 30, 2002 made pursuant to the securities laws, and for other purposes''  \citep[1]{United-States-Government:2002vn}, the Sarbanes--Oxley Act brought several new regluations such as that the CEO and CFO are required to certify the accuracy of corporate financial reports which have published timely along with reports on the effectiveness of the company's internal control structures and financial reporting procedures.}}

\newglossaryentry{glos:sme}{
name=Small and Medium Enterprise,
description={Company whose personnel number falls under
certain thresholds, typically in the range of maximum 100--250
people, depending on the jurisdiction.}}

\def\descstakeholder{According to \citeauthor{Donaldson:1995uq}, anyone
who has a ``stake, (i.e., potential benefit)''  (\citeyear[86]{Donaldson:1995uq}), in the 
company, i.e., they consider
the Stanford Research Institute's definition of
stakeholders as ``those groups without whose support the organization would cease to exist''
(\citeyear[72]{Donaldson:1995uq}) as too narrow,  and while they hold that ``it is
essential to draw a clear line between influencers and stakeholders,''
(\citeyear[loc. cit.]{Donaldson:1995uq}) they see managers not only as
``referees'' between the two stakeholder groups of investors and employees, but emphasize
``the fact that managers of a firm are one of its most important and powerful
constituencies.'' (\citeyear[loc.
cit.]{Donaldson:1995uq}) The analysis that
\citeauthor{Donaldson:1995uq} take is oriented towards a normative argument 
which works out the responsibility of managers in that they ``\emph{should}
acknowledge diverse stakeholder interests and  \emph{should} attempt to respond
to them within a mutually supportive framework because that's a moral
requirement for the legitimacy of the management function.'' 
(\citeyear[p. 87, emphasis theirs]{Donaldson:1995uq}).}
\newglossaryentry{glos:stakeholder}{
name=stakeholder,
firstplural=stakeholders,
description={\descstakeholder}}

\newglossaryentry{glos:agencytheory}{
name=Agency Theory,
description={According to \citeauthor{Kolb:2008fk},
``the theory of agency seeks to explain why and how service and control can succeed or fail \dots One actor, the agent, is modeled as acting for another, the principal. \dots The actor's problems focus on serving
[or not] the principal; the principal's problems generally entail dilemmas of how to assure that the agent will do what the principal wants him or her to do.'' (\citeyear[42]{Kolb:2008fk})}}

\def\glosdescifrs{International Financial Reporting Standards (IFRS) are designed as a common global language for business affairs so that company accounts are understandable and comparable across international boundaries. They are a consequence of growing international shareholding and trade and are particularly important for companies that have dealings in several countries. They are progressively replacing the many different national accounting standards.The rules to be followed by accountants to maintain books of accounts which is comparable, understandable, reliable and relevant as per the users internal or external.}
\newglossaryentry{glos:ifrs}{
  name=International Financial Reporting Standard (IFRS),
  firstplural=International Financial Reporting Standards (IFRS),
  plural=International Financial Reporting Standards,
  description={\glosdescifrs}
}

\def\glosdescgaap{Generally accepted accounting principles (GAAP) refer to the standard framework of guidelines for financial accounting used in any given jurisdiction; generally known as accounting standards or standard accounting practice. These include the standards, conventions, and rules that accountants follow in recording and summarizing and in the preparation of financial statements.}
\newglossaryentry{glos:gaap}{
  name=Generally accepted accounting principles (GAAP),
  firstplural=Generally accepted accounting principles (GAAP),
  plural=Generally accepted accounting principles (GAAP),
  description={\glosdescgaap}
}

\def\glosdescxbrl{eXtensible Business Reporting Language (XBRL) is a standards-based way to communicate and exchange business information between business systems. These communications are defined by metadata set out in taxonomies, which capture the definition of individual reporting concepts as well as the relationships between concepts and other semantic meaning.}
\newglossaryentry{glos:xbrl}{
  name=Extensible Business Reporting Languages (XBRL),
  firstplural=eXtensible Business Reporting Languages (XBRL),
  plural=eXtensible Business Reporting Languages (XBRL),
  description={\glosdescxbrl}
}

\def\glosdescaccrualsaccounting{``Accruals accounting is preparing the income statement and statement of financial position following the \gls{glos:accrualsconvention}, which says that profit = revenue - expenses (not cash receipts - cash payments).'' \citep[107]{Atrill:2011qy}}
\newglossaryentry{glos:accrualsaccounting}{
  name=accruals accounting,
  firstplural=accruals accountings,
  plural=accruals accountings,
  description={\glosdescaccrualsaccounting}
}


\def\glosdescaccrualsconvention{``The convention of accounting that asserts that profit is the excess of revenue over expenses, 
not the excess of cash receipts over cash payments.'' \citep[632]{Atrill:2010ys}}
\newglossaryentry{glos:accrualsconvention}{
  name=accruals convention,
  firstplural=accruals conventions,
  plural=accruals conventions,
  description={\glosdescaccrualsconvention}
}

\def\glosdescliquidity{Liquidity is the ability to pay short-term obligations
when they fall due. ``It is vital to the survival of a business that there are
sufficient liquid resources available to meet maturing obligations (that is,
amounts owing that must be paid in the near future).''
\citep[141]{Atrill:2010ys}}
\newglossaryentry{glos:liquidity}{
  name=liquidity,
  firstplural=liquidities,
  plural=liquidities,
  description={\glosdescliquidity}
}

\def\glosdescprofitability{The quality of affording gain or profit. ``Businesses
generally exist with the primary purpose of creating wealth for their owners.
Profitability ratios provide an insight to the degree of success in achieving
this purpose. They express the profit made (or figures bearing on profit, such
as sales revenue or particular expenses, like labour cost) in relation to other
key figures in the financial statements or to some business resource.'' \citep[141]{Atrill:2010ys}}
\newglossaryentry{glos:profitability}{
  name=profitability,
  firstplural=profitabilities,
  plural=profitabilitie	s,
  description={\glosdescprofitability}
}



%</content>%























