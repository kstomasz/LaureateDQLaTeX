%=====================================================
% Document Type and Geometry
%=====================================================

\documentclass[12pt,a4paper,fleqn,twoside]{\doctype}	  	
%\write18{if [ ! -f tmp/wc.tex -o ! -s tmp/wc.tex ] ; then make wc verbose=0 loglvl=error fromtex=1 > tmp/wc.tex ; fi}
\input{tmp/env.tex}

%=====================================================
% Some Helpers
%=====================================================

\usepackage{morewrites}								% More write space
\usepackage{lipsum}										% Lorem Impsum...
\usepackage{xparse}										% Parsing of Parameters
\usepackage{xifthen}										% If Then Else
\usepackage{ifpdf}										% Generating PDF or not
\ifpdf\usepackage{cmap}\fi							% Makes PDF searchable, for Skim
\usepackage{xstring}										% String functions
\usepackage[titles]{tocloft}							% New Listofs...

%=====================================================
% Document Geometry
%=====================================================

\usepackage[hmarginratio=1:1]{geometry}
\geometry{
  left=2.5cm,													% left margin
  top=3.5cm,													% top  margin
  textwidth=16cm,											% width of text block
  textheight=21.0cm}										% height of text block
\setlength{\headheight}{1cm}						% height of header
\setlength{\headsep}{1cm}							% distance of header
\setlength{\footskip}{1.5cm}						% distance of footer
\renewcommand{\baselinestretch}{1.0}		% 1line distance
\clubpenalty10000											% Don't want clubss
\widowpenalty10000										% Don't want widows
\hbadness 10000												% Reduce underfull hbox warnings
\sloppy															% Allow sloppy layout, reduce hyphenations


%=====================================================
% Language
%=====================================================

\usepackage[american]{babel}						% Internationalization
\usepackage[utf8]{inputenc}							% Allow for utf-8 in input files
%\usepackage[T1]{fontenc}							% Breaks utf-8 in bibliography							
\usepackage{csquotes}  								% Extended quoting
\selectlanguage{american}							% Select English


%=====================================================
% Fixups for HTML
%=====================================================

\ifthenelse{\equal{\doctype}{book}}{
\ifpdf
\else
  \renewcommand\markboth{\null}			% Markboth makes no sense in HTML
\fi
}{}

\ifpdf
	\usepackage[english]{varioref}					% vref and related	
\else
	\usepackage{tex4ht}
   \newcommand\vref{\ref}							% varioref doesn't work in HTML
   \newcommand\textwidth{\linewidth}
\fi


%=====================================================
% Word counting
%=====================================================

\ifpdf
\else
  \def\HCode#1{}											% Allow plain HTML output
\fi
\newcommand\wcounta{\ifpdf\else\HCode{<!-- COUNT -->}\fi}
\newcommand\wcounte{\ifpdf\else\HCode{<!-- /COUNT -->}\fi }


%=====================================================
% Headers
%=====================================================

\usepackage{fancyhdr}									% Fancy Page Headers
\usepackage[Sonny]{fncychap}					% Fancy Chapter Headers
\ChNumVar{\fontsize{60}{62}}					% Set Font Size for Chapter Headers


%=====================================================
% Floats, Tables, Equations, etc.
%=====================================================

\usepackage{boxedminipage}						% Minipage with a box around it
\usepackage{wrapfig}									% Allow for floats wrapped by text
\usepackage[hang]{caption}							% captions with hanging indent
\usepackage{capt-of} 									% caption{figure}[]{}, outside float
\usepackage[section,above] {placeins}			% Prevent floats floating past the section
\usepackage{float}											% Improved control over floats
\usepackage{booktabs}									% Better tables
\usepackage{adjustbox}								% Flexible boxes, color, rotate, etc.
\usepackage{fancybox}									% Boxes, also for equations
\usepackage{array}										% Better tabular and array environments
\usepackage{colortbl}									% Colored Tables
\usepackage{color}										% Color functionality
\usepackage[table]{xcolor}							% Extended color functionality (blue!20 etc.)


%
% Alter some LaTeX defaults for better treatment of figures:
% See p.105 of "TeX Unbound" for suggested values.
% See pp. 199-200 of Lamport's "LaTeX" book for details.
%  
% General parameters, for ALL pages:
%
\renewcommand{\topfraction}{0.9}			% max fraction of floats at top
\renewcommand{\bottomfraction}{0.8}		% max fraction of floats at bottom
%
% Parameters for TEXT pages (not float pages):
%
\setcounter{topnumber}{2}							% max floats at top of page
\setcounter{bottomnumber}{2}					% max floats at bottom of page
\setcounter{totalnumber}{4}     					% max floats on page
\setcounter{dbltopnumber}{2}    					% for 2-column pages
\renewcommand{\dbltopfraction}{0.9}		% fit big float above 2-col. text
\renewcommand{\textfraction}{0.07}		% allow minimal text w. figs
%
% Parameters for FLOAT pages (not text pages):
%
\renewcommand{\floatpagefraction}{0.7}	% require fuller float pages
% N.B.: floatpagefraction MUST be less than topfraction !!
\renewcommand{\dblfloatpagefraction}{0.7}% require fuller float pages
% Remember to use [htp] or [htpb] for placement


%=====================================================
% Code Listings
%=====================================================

\usepackage{listings}
\definecolor{lstemph}{rgb}{0,0.39,0} 
\definecolor{lstnumbers}{rgb}{0.59,0.57,0.43} 
\definecolor{lstcomments}{rgb}{0.33,0.35,0.69} 
\lstloadlanguages{Java,C++}
\lstset{language=Java,
		extendedchars=true,
        basicstyle=\ttfamily\tiny,
        keywordstyle=\color{lstnumbers},
        identifierstyle=\color{black}, 
        commentstyle=\color{lstcomments},
        stringstyle=\ttfamily\color{blue},
        showstringspaces=true,
        numbers=left,
        stepnumber=1,
        numberstyle=\tiny\ttfamily\color{lstnumbers},
        numbersep=12pt,
        frame=single,
        fontadjust=true,
        xleftmargin=3.5pt,
        xrightmargin=3.5pt,
        escapeinside={(*}{*)}}


%=====================================================
% Hyperref
%=====================================================

\ifpdf
\usepackage[%pdftex,
            pdfpagemode={UseOutlines},
            pdfstartview={FitH},
            colorlinks=true,   
            linkcolor={blue},
            citecolor={blue}, 
            urlcolor={blue},
            bookmarks=true,
            bookmarksopen=true,
            %pdfpagemode=FullScreen,
            %hyperindex=false,
            plainpages=false,
            %hypertexnames=false,
            pdfpagelabels]{hyperref} 
\else
\usepackage[tex4ht]{hyperref}
\fi


%=====================================================
% Massively ugly workaround of hyperrefs issues with equations in glossaries
%=====================================================

%===<exclude> for word count

\ifpdf
  \makeatletter
  \renewcommand{\theHequation}{\@currentHref.\arabic{equation}}
  \gdef\equationgrouping{}
  \makeatother
\fi

%===</exclude>


%=====================================================
% Glossaries
%=====================================================

\usepackage[
nonumberlist, 													% do not show page numbers
acronym,        													% generate acronym listing
toc,                													% show listings as entries in table of contents
section]         													% use section level for toc entries
{glossaries}

%
% Patch Glossaries so that only the first occurrence of a given glossary
% entry is converted to a hyperlink, in order to avoid cluttering.
%
\makeatletter
%% patch first occurence of "\@gls@link[#1]{#2}{\@glo@text}", 
%% as this is the one for \glsused{#2}
\patchcmd{\@gls@}
  {\@gls@link[#1]{#2}{\@glo@text}}
  {\@gls@link[#1,hyper=false]{#2}{\@glo@text}}
  {}{}
\patchcmd{\@glspl@}
  {\@gls@link[#1]{#2}}
  {\@gls@link[#1,hyper=false]{#2}}
  {}{}
\patchcmd{\@Gls@}
  {\@gls@link[#1]{#2}}
  {\@gls@link[#1,hyper=false]{#2}}
  {}{}
\patchcmd{\@Glspl@}
  {\@gls@link[#1]{#2}}
  {\@gls@link[#1,hyper=false]{#2}}
  {}{}
  \patchcmd{\@GLS@}
  {\@gls@link[#1]{#2}{\MakeUppercase{\@glo@text}}}
  {\@gls@link[#1,hyper=false]{#2}{\MakeUppercase{\@glo@text}}}
  {}{}  
\makeatother

%
% Generate a list of symbols
%
\newglossary[slg]{symbolslist}{syi}{syg}{List of Symbols}

%
% Make sure first character of glossary entry name is uppercase
%
\renewcommand{\glsnamefont}[1]{\makefirstuc{#1}}

%
% Remove the dot at the end of glossary descriptions
%
\renewcommand*{\glspostdescription}{}

%
% Activate glossary commands
%
\makeglossaries

%
% Load the glossary definitions the user writes
%
%=====================================================
% Acronyms
%=====================================================

%-----------------------------------------------------
% Samples
%-----------------------------------------------------

% Usage:
% \gls{AD} is pretty interesting. If we do reference a glossary entry,
% like, \gls{glos:AD}, that one of course has to be defined over there.

% An acronym with a glossary entry not hyperlinked
% \newacronym{AD}{AD}{Active Directory\protect\glsadd{glos:AD}}

% Geek Version, with hyperlink to glossary
% \newglossaryentry{AD}{
%   type=\acronymtype,
%   name=AD,
%   first=Active Directory (AD),
%   firstplural={Active Directories (AD's)},
%   description=\glslink{glos:AD}{Active Directory}
% }

%----------------------------------------------------- 
% Content
%-----------------------------------------------------

%<content>%



%</content>%


%-----------------------------------------------------
% Default Content
%-----------------------------------------------------

    
 
   
 
 
 
  
  








































%=====================================================
% Symbols
%=====================================================

%-----------------------------------------------------
% Samples
%-----------------------------------------------------

% Usage:
% \section{Some Greek symbols}
% If you calculate with \gls{symb:Pi} you always get an irrational result, because 
% \gls{symb:Pi} itself is irrational. As a matter of fact, there are \gls{symb:Phi} 
% and \gls{symb:Lambda}, too.

%Some entries for the list of symbols
\newglossaryentry{symb:Pi}{
  type=symbolslist,
  name=$\pi$,
  description={A mathematical constant whose value is the ratio of any circle's circumference to its diameter.},
  sort=symbolpi
}
\newglossaryentry{symb:Phi}{
  type=symbolslist,
  name=$\varphi$,
  description={An angle.},
  sort=symbolphi
}
\newglossaryentry{symb:Lambda}{
  type=symbolslist,
  name=$\lambda$,
  description={Lambda indictes an eigenvalue in the mathematics  of linear algebra.},
  sort=symbollambda
}

%-----------------------------------------------------
% Content
%-----------------------------------------------------

%<content>%



%</content>%



























%=====================================================
% Glossary
%=====================================================

%-----------------------------------------------------
% Samples
%-----------------------------------------------------

% Usage:
% \gls{glos:AD} is pretty interesting. If we have a cross reference from
% the acronyms, we can also directly go to that using \gls{AD}; this
% requires then that over there, we have something like
%  see=[Glossary:]{\gls{glos:AD}}, 
%  description=\glslink{glos:AD}{Active Directory}

% \newglossaryentry{glos:AD}{
% name=Active Directory,
% description={Active Directory is the directory service for 
% Windows based networks, that allows central organization and 
% administration of any network resource.
% It allows a single-sign-on concept independent from network 
% topologies or network protocols. As a prerequisite you need 
% a Windows Server acting as Domain Controller. This computer 
% stores all necessary data, e.\,g.~usernames and corresponding 
% passwords.}
% }


%-----------------------------------------------------
% Content
%-----------------------------------------------------

%<content>%



%</content>%















%
% These commands actually create / update the different
% indices / glossaries
%
%makeindex -s document.ist -t document.alg -o document.acr document.acn
%makeindex -s document.ist -t document.glg -o document.gls document.glo
%makeindex -s document.ist -t document.slg -o document.syi document.syg
%makeindex document


%=====================================================
% Index
%=====================================================

\usepackage{makeidx}


%=====================================================
% Graphics
%=====================================================
\usepackage{graphicx}
\ifpdf
  \graphicspath{{pdf/}}
  \pdfcompresslevel=9
  \DeclareGraphicsExtensions{.pdf}
  \DeclareGraphicsRule{.pdf}{pdf}{.pdf}{}
\else
  \graphicspath{{eps/}}
  \DeclareGraphicsExtensions{.eps}
  \DeclareGraphicsRule{.eps}{eps}{.eps}{}
\fi


%=====================================================
% Media
%=====================================================

%\renewcommand{\video}[6]{% file xpos ypos width height controls
%  \vspace{#3}\hspace{#2}{\pdfannot width #4 height #5 depth 0cm {%
%   /Subtype /Movie  
%   /Movie  << /F (#1) >> 
%   /A << /ShowControls #6 /Rate 1 >>
%   }}}
%\fi

\usepackage{sty/easymovie/easymovie}


%=====================================================
% Equations
%=====================================================

\usepackage[fleqn,tbtags]{mathtools}			% Mathematical Processing
\usepackage{amssymb}									% Scientific Symbols
\usepackage{latexsym}									%	Scientific Symbols
\mathtoolsset{showonlyrefs}						% Label only (eqref) referenced Equations 
%\everymath{\rm}											% Default to roman style

\usepackage[customcolors]{hf-tikz}				% Highlight Formulas
\usetikzlibrary{calc}
\tikzstyle{every picture}+=[remember picture]
\hfsetfillcolor{blue!10}
\hfsetbordercolor{blue}


%=====================================================
% Font Settings
%=====================================================

\usepackage{microtype}								% More precise typography
\usepackage{fix-cm}										% Permit arbitrary font sizes
\usepackage{eurosym}									% Euro Symbol

%
% Specify fonts for Captions, Sectionts, etc.
%
\renewcommand*\captionlabelfont{\bfseries}	
\renewcommand*\captionsize{\itshape}		

\makeatletter
\renewcommand{\section}{\@startsection{section}{1}{\z@}%
    {-2.2ex \@plus-1ex \@minus -.2ex}{1.3ex \@plus.2ex}%
    {\reset@font\large\bfseries}}
\renewcommand{\subsection}{\@startsection{subsection}{2}{\z@}%
    {-1.5ex \@plus -1ex \@minus-.2ex}{0.8ex \@plus.2ex}%
    {\reset@font\normalsize\bfseries}}
\renewcommand{\subsubsection}{\@startsection{subsubsection}{3}{\z@}%
     {-1.2ex\@plus -1ex \@minus -.2ex}{0.5ex \@plus .2ex}%
     {\reset@font\normalsize}}
 \renewcommand{\paragraph}{\@startsection{paragraph}{4}{0mm}%
  {1ex \@plus1ex \@minus.2ex}%
  {-1em}%
  {\normalfont\normalsize\it}}
 \renewcommand{\subparagraph}{\@startsection{subparagraph}{5}{\parindent}%
  {2.0ex \@plus1ex \@minus .2ex}%
  {-1em}%
  {\normalfont\normalsize\it}}     
\makeatother

%
% Itemizes
%
\renewcommand{\labelitemi}{$\triangleright$}
\renewcommand*\descriptionlabel[1]{\hspace\labelsep
                                \normalfont\itshape #1}

%
% Save Default Space above Itemize etc., then set it to 0
%
% Not resetting, need further investigation.
%
% \newlength{\oldabovedisplayskip}
% \setlength{\oldabovedisplayskip}{\abovedisplayskip}
% %\setlength{\abovedisplayskip}{0pt}
% \expandafter\def\expandafter\normalsize\expandafter{%
% \normalsize\setlength\oldabovedisplayskip{\abovedisplayskip}}
% \expandafter\def\expandafter\normalsize\expandafter{%
% \normalsize\setlength\abovedisplayskip{0pt}}
% \expandafter\def\expandafter\normalsize\expandafter{%
% \normalsize\setlength\abovedisplayskip{\oldabovedisplayskip}}


%=====================================================
% Table of... / Listofs
%=====================================================

\usepackage[titles]{tocloft}							% New Listofs...
\setcounter{secnumdepth}{10}						% Section numbers down to level 10
\setcounter{tocdepth}{3}								% TOC content down to level 3

%
% Give some more room for page numbers
%
\makeatletter
\renewcommand{\@pnumwidth}{3em} 
\renewcommand{\@tocrmarg}{4em}
\makeatother


%=====================================================
% Footnotes / Endnotes
%=====================================================

\usepackage[flushmargin, hang]{footmisc}	% More footnote options
\usepackage{sty/botfnote/botfnote}				% Force footnotes to the bottom
\usepackage[backref,counter-format=arabic]{sty/enotez/enotez} % Backreferencing Endnotes
 
\ifpdf
  \usepackage{sty/hyperendnote/hyperendnote} %  Referenced Endnotes
\fi


\usepackage{setspace}

\DeclareInstance{enotez-list}{itemize}{list}{
  list-type = itemize,
  number = \enmark{#1} ,
  format = \footnotesize,
}

%=====================================================
% Load Macros
%=====================================================    

%=====================================================
% Makros
%=====================================================

%-----------------------------------------------------
% Simple Substitutions:
%-----------------------------------------------------

\def\ni{\noindent}
\def\usw{$[\dots]$}
\def\daher{$\rightarrow$}
\def\tab{\hspace{2 cm}}
\def\fn{\footnote}
\def\en{\endnote}
\def\Unterschrift{\newline \includegraphics[width=4cm]{unter} \newline}
\def\unterschrift{\Unterschrift}
\newcommand{\bs}{$\backslash$}
 \def\LRA{\Leftrightarrow\mkern40mu}
 \def\RA{\Rightarrow\mkern40mu}
 
%-----------------------------------------------------
% dangerous / ddangerous etc. environments a la Knuth: 
%-----------------------------------------------------

\font\manual=manfnt
\def\dbend{{\manual\char127}}
\def\d@nger{\medbreak\begingroup\clubpenalty=10000
  \def\par{\endgraf\endgroup\medbreak}  
\noindent\hang\hangafter=-2  \hbox 
to0pt{\hskip-\hangindent\dbend\hfill}\ninepoint}
\outer\def\danger{\d@nger}
\def\dd@nger{\medbreak\begingroup\clubpenalty=10000
  \def\par{\endgraf\endgroup\medbreak} 
\noindent\hang\hangafter=-2  \hbox 
to0pt{\hskip-
\hangindent\dbend\kern1pt\dbend\hfill}\ninepoint} 
\outer\def\ddanger{\dd@nger}
\def\enddanger{\endgraf\endgroup}

\newsavebox{\fmbox}
\newenvironment{notdangerous}
{
 \begin{lrbox}{\fmbox}
 \begin{minipage}[t]{1cm}~\end{minipage}
 \begin{minipage}[t]{1.5cm}\hspace{\fill}~\end{minipage}
 \begin{minipage}[t]{0.1cm}~\end{minipage}
 \begin{minipage}[t]{12.5cm}
}
{\end{minipage}\end{lrbox}\usebox{\fmbox}
}
\newenvironment{dangerous}
{
 \begin{lrbox}{\fmbox}
 \begin{minipage}[t]{0.2cm}~\end{minipage}
 \begin{minipage}[t]{1.5cm}\hspace{\fill}\dbend\end{minipage}
 \begin{minipage}[t]{0.1cm}~\end{minipage}
 \begin{minipage}[t]{10.8cm}
}
{\end{minipage}\end{lrbox}\usebox{\fmbox}
}
\newenvironment{ddangerous}
{
 \begin{lrbox}{\fmbox}
 \begin{minipage}[t]{0.2cm}~\end{minipage}
 \begin{minipage}[t]{1.5cm}\hspace{\fill}\dbend\dbend\end{minipage}
 \begin{minipage}[t]{0.1cm}~\end{minipage}
 \begin{minipage}[t]{10.8cm}
}
{\end{minipage}\end{lrbox}\usebox{\fmbox}
}
\newenvironment{dddangerous}
{
 \begin{lrbox}{\fmbox}
 \begin{minipage}[t]{0.2cm}~\end{minipage}
 \begin{minipage}[t]{1.5cm}\hspace{\fill}\dbend\dbend\dbend\end{minipage}
 \begin{minipage}[t]{0.1cm}~\end{minipage}
 \begin{minipage}[t]{10.8cm}
}
{\end{minipage}\end{lrbox}\usebox{\fmbox}
}

%-----------------------------------------------------
% Other environments:
%-----------------------------------------------------

\newcommand{\companyfooter} {
\vspace*{0.2cm}
\setlength{\tabcolsep}{0.05cm}
\tiny
\centerline{
\begin{tabular}{p{1.5cm}p{4.5cm}p{6.0cm}p{0.5cm}p{2.96cm}}
\toprule[0.25pt]
\makebox[1.50cm][l]{Module:} &
\makebox[4.50cm][l]{\mnbook}&
\makebox[6.00cm][c]{\mnname}&
\makebox[0.50cm][l]{$\:$Date:}&
\makebox[2.96cm][r]{\today}
\\%\midrule[0.15pt]
\makebox[1.50cm][l]{Document:} &
\makebox[4.50cm][l]{\mnsubtitle}&
\makebox[6.00cm][c]{\mnsubsubtitle\ (\mnmoduleweek)}&
\makebox[0.50cm][l]{$\:$Version:}&
\makebox[2.96cm][r]{\mnversion}
\\
\bottomrule[0.25pt]
\end {tabular}
}}

