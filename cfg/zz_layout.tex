%-----------------------------------------------------
% Overall Geometry of the Document
%-----------------------------------------------------
\documentclass[12pt,a4paper,fleqn,twoside]{article}	  	
\write18{if [ ! -f tmp/wc.tex -o ! -s tmp/wc.tex ] ; then make wc verbose=0 silent=1 > tmp/wc.tex ; fi}
\usepackage{cmap}

\usepackage[hmarginratio=1:1]{geometry}
\geometry{%
  left=2.5cm,				% left margin
  top=3.5cm,				% top  margin
  textwidth=16cm,		% width of text block
  textheight=21.0cm}	% height of text block
\setlength{\headheight}{1cm}					    % height of header
\setlength{\headsep}{1cm}							% distance of header
\setlength{\footskip}{1.5cm}					    % distance of footer
\renewcommand{\baselinestretch}{1.0}     % 1line distance

%----------------------------------------------------- 
% Language
%-----------------------------------------------------
\usepackage[english]{varioref}			  
\usepackage[american]{babel}
\usepackage[utf8]{inputenc}
\usepackage[T1]{fontenc}
\usepackage{csquotes}  
\selectlanguage{american}


%-----------------------------------------------------
% Headers
%-----------------------------------------------------
\usepackage{fancyhdr}
\usepackage[Sonny]{sty/fncychap/fncychap}
\ChNumVar{\fontsize{60}{62}}


%-----------------------------------------------------
% Floats, Tables, Equations
%-----------------------------------------------------
\usepackage[hang]{caption} 
\usepackage[section] {placeins}
\usepackage{float}
\usepackage{colortbl}
\usepackage{booktabs}
\usepackage{latexsym}
\setcounter{totalnumber}{4}                     % max. no. of floats / page
\setcounter{topnumber}{2}                       % max. no. of floats / page top
\setcounter{bottomnumber}{2}                 % max. no. of floats / page bottom
\renewcommand{\topfraction}{1}           % half the page can be filled with
\renewcommand{\bottomfraction}{1}    % floats for top and bottom half
\usepackage{wrapfig}

%-----------------------------------------------------
% Listings
%-----------------------------------------------------
\usepackage{listings}
\definecolor{lstemph}{rgb}{0,0.39,0} 
\definecolor{lstnumbers}{rgb}{0.59,0.57,0.43} 
\definecolor{lstcomments}{rgb}{0.33,0.35,0.69} 
\lstloadlanguages{Java,C++}
\lstset{language=Java,
		extendedchars=true,
        basicstyle=\ttfamily\tiny,
        keywordstyle=\color{lstnumbers},
        identifierstyle=\color{black}, 
        commentstyle=\color{lstcomments},
        stringstyle=\ttfamily\color{blue},
        showstringspaces=true,
        numbers=left,
        stepnumber=1,
        numberstyle=\tiny\ttfamily\color{lstnumbers},
        numbersep=12pt,
        frame=single,
        fontadjust=true,
        xleftmargin=3.5pt,
        xrightmargin=3.5pt,
        escapeinside={(*}{*)}}

%-----------------------------------------------------
% Processing of the document
%-----------------------------------------------------
% Includex is for combining documents.
% It is obsolete and throws warnings,
% so we do not use it as long as we do not
% need it. See:
%
% http://www.tex.ac.uk/cgi-bin/texfaq2html?label=multidoc
% 
%\usepackage{sty/includex/includex}

%-----------------------------------------------------
% Hyperref
%-----------------------------------------------------
%\ifx\pdfoutput\undefined
%\usepackage[ps2pdf]{hyperref}
%\else
\usepackage[%pdftex,
            pdfpagemode={UseOutlines},
            pdfstartview={FitH},
            colorlinks=true,   
            linkcolor={blue},
            citecolor={blue}, 
            urlcolor={blue},
            bookmarks=true,
            bookmarksopen=true,
            %pdfpagemode=FullScreen,
            plainpages=false,
            pdfpagelabels]{hyperref} 

%-----------------------------------------------------
% Nomenclature / Glossary
%-----------------------------------------------------

%\usepackage[
%  style=altlist,
%  hypertoc=true,
%   hyper=true,
%   number=none,
%   hyperacronym=true,
%   acronym=true %dieser Parameter ist der wichtige
% ]{sty/glossary/glossary}
% \setacronymnamefmt{\gloshort}
% \makeglossary
% \makeacronym

\usepackage[
nonumberlist, %do not show page numbers
acronym,        %generate acronym listing
toc,                %show listings as entries in table of contents
section]         %use section level for toc entries
{glossaries}

%
% Patch Glossaries so that only the first occurrence of a given glossary entry
% is converted to a hyperlink, in order to avoid cluttering.
%
\makeatletter
%% patch first occurence of "\@gls@link[#1]{#2}{\@glo@text}", as this is the one for \glsused{#2}
\patchcmd{\@gls@}
  {\@gls@link[#1]{#2}{\@glo@text}}
  {\@gls@link[#1,hyper=false]{#2}{\@glo@text}}
  {}{}
\patchcmd{\@glspl@}
  {\@gls@link[#1]{#2}}
  {\@gls@link[#1,hyper=false]{#2}}
  {}{}
\patchcmd{\@Gls@}
  {\@gls@link[#1]{#2}}
  {\@gls@link[#1,hyper=false]{#2}}
  {}{}
\patchcmd{\@Glspl@}
  {\@gls@link[#1]{#2}}
  {\@gls@link[#1,hyper=false]{#2}}
  {}{}
  \patchcmd{\@GLS@}
  {\@gls@link[#1]{#2}{\MakeUppercase{\@glo@text}}}
  {\@gls@link[#1,hyper=false]{#2}{\MakeUppercase{\@glo@text}}}
  {}{}  
\makeatother

%
% Generate a list of symbols
%
\newglossary[slg]{symbolslist}{syi}{syg}{List of Symbols}

%
% Make sure first character of glossary entry name is uppercase
%
\renewcommand{\glsnamefont}[1]{\makefirstuc{#1}}

%
% Remove the dot at the end of glossary descriptions
%
\renewcommand*{\glspostdescription}{}

%
% Activate glossary commands
%
\makeglossaries

%
% Load the glossary definitions the user writes
%
%=====================================================
% Acronyms
%=====================================================

%-----------------------------------------------------
% Samples
%-----------------------------------------------------

% Usage:
% \gls{AD} is pretty interesting. If we do reference a glossary entry,
% like, \gls{glos:AD}, that one of course has to be defined over there.

% An acronym with a glossary entry not hyperlinked
% \newacronym{AD}{AD}{Active Directory\protect\glsadd{glos:AD}}

% Geek Version, with hyperlink to glossary
% \newglossaryentry{AD}{
%   type=\acronymtype,
%   name=AD,
%   first=Active Directory (AD),
%   firstplural={Active Directories (AD's)},
%   description=\glslink{glos:AD}{Active Directory}
% }

%----------------------------------------------------- 
% Content
%-----------------------------------------------------

%<content>%



%</content>%


%-----------------------------------------------------
% Default Content
%-----------------------------------------------------

    
 
   
 
 
 
  
  








































%=====================================================
% Symbols
%=====================================================

%-----------------------------------------------------
% Samples
%-----------------------------------------------------

% Usage:
% \section{Some Greek symbols}
% If you calculate with \gls{symb:Pi} you always get an irrational result, because 
% \gls{symb:Pi} itself is irrational. As a matter of fact, there are \gls{symb:Phi} 
% and \gls{symb:Lambda}, too.

%Some entries for the list of symbols
\newglossaryentry{symb:Pi}{
  type=symbolslist,
  name=$\pi$,
  description={A mathematical constant whose value is the ratio of any circle's circumference to its diameter.},
  sort=symbolpi
}
\newglossaryentry{symb:Phi}{
  type=symbolslist,
  name=$\varphi$,
  description={An angle.},
  sort=symbolphi
}
\newglossaryentry{symb:Lambda}{
  type=symbolslist,
  name=$\lambda$,
  description={Lambda indictes an eigenvalue in the mathematics  of linear algebra.},
  sort=symbollambda
}

%-----------------------------------------------------
% Content
%-----------------------------------------------------

%<content>%



%</content>%



























%=====================================================
% Glossary
%=====================================================

%-----------------------------------------------------
% Samples
%-----------------------------------------------------

% Usage:
% \gls{glos:AD} is pretty interesting. If we have a cross reference from
% the acronyms, we can also directly go to that using \gls{AD}; this
% requires then that over there, we have something like
%  see=[Glossary:]{\gls{glos:AD}}, 
%  description=\glslink{glos:AD}{Active Directory}

% \newglossaryentry{glos:AD}{
% name=Active Directory,
% description={Active Directory is the directory service for 
% Windows based networks, that allows central organization and 
% administration of any network resource.
% It allows a single-sign-on concept independent from network 
% topologies or network protocols. As a prerequisite you need 
% a Windows Server acting as Domain Controller. This computer 
% stores all necessary data, e.\,g.~usernames and corresponding 
% passwords.}
% }


%-----------------------------------------------------
% Content
%-----------------------------------------------------

%<content>%



%</content>%















%
% These commands actually create / update the different
% indices / glossaries
%
%makeindex -s document.ist -t document.alg -o document.acr document.acn
%makeindex -s document.ist -t document.glg -o document.gls document.glo
%makeindex -s document.ist -t document.slg -o document.syi document.syg
%makeindex document

%-----------------------------------------------------
% Index
%-----------------------------------------------------

\usepackage{makeidx}

%-----------------------------------------------------
% Media
%-----------------------------------------------------

%\renewcommand{\video}[6]{% file xpos ypos width height controls
%  \vspace{#3}\hspace{#2}{\pdfannot width #4 height #5 depth 0cm {%
%   /Subtype /Movie  
%   /Movie  << /F (#1) >> 
%   /A << /ShowControls #6 /Rate 1 >>
%   }}}
%\fi

\usepackage{sty/easymovie/easymovie}


%-----------------------------------------------------
% Equations
%-----------------------------------------------------

\usepackage[fleqn,tbtags]{mathtools}
\mathtoolsset{showonlyrefs}

%
% Default to roman style
%
%\everymath{\rm}





%-----------------------------------------------------
% Font Settings
%-----------------------------------------------------
\renewcommand*\captionlabelfont{\bfseries}	
\renewcommand*\captionsize{\itshape}		

\makeatletter
\renewcommand{\section}{\@startsection{section}{1}{\z@}%
    {-2.2ex \@plus-1ex \@minus -.2ex}{1.3ex \@plus.2ex}%
    {\reset@font\large\bfseries}}
\renewcommand{\subsection}{\@startsection{subsection}{2}{\z@}%
    {-1.5ex \@plus -1ex \@minus-.2ex}{0.8ex \@plus.2ex}%
    {\reset@font\normalsize\bfseries}}
\renewcommand{\subsubsection}{\@startsection{subsubsection}{3}{\z@}%
     {-1.2ex\@plus -1ex \@minus -.2ex}{0.5ex \@plus .2ex}%
     {\reset@font\normalsize}}
 \renewcommand{\paragraph}{\@startsection{paragraph}{4}{0mm}%
  {1ex \@plus1ex \@minus.2ex}%
  {-1em}%
  {\normalfont\normalsize\it}}
 \renewcommand{\subparagraph}{\@startsection{subparagraph}{5}{\parindent}%
  {2.0ex \@plus1ex \@minus .2ex}%
  {-1em}%
  {\normalfont\normalsize\it}}     
\makeatother

%-----------------------------------------------------
% Itemizes etc.
%-----------------------------------------------------
\renewcommand{\labelitemi}{$\triangleright$}
\renewcommand*\descriptionlabel[1]{\hspace\labelsep
                                \normalfont\itshape #1}

%-----------------------------------------------------
% Widow etc. compensations
%-----------------------------------------------------
\clubpenalty10000					
\widowpenalty10000					
\hbadness 10000
\scrollmode
\sloppy


%-----------------------------------------------------
% Depth of Table of Contents
%-----------------------------------------------------
\setcounter{secnumdepth}{10}				
\setcounter{tocdepth}{3}

%-----------------------------------------------------
% PDF or not
%-----------------------------------------------------
\usepackage{ifpdf}


%-----------------------------------------------------
% Footnotes
%-----------------------------------------------------
\usepackage[flushmargin, hang]{footmisc}		
\usepackage{sty/botfnote/botfnote}
%\usepackage{endnotes}
\usepackage[backref]{sty/enotez/enotez}

%Disable to have actual footnotes
%\let\footnote=\endnote
%\let\footnotemark=\endnotemark
%\let\footnotetext=\endnotetext
 %remember to substitute \value{footnote} by \endnote or vice versa
 
\ifpdf
  \usepackage{sty/hyperendnote/hyperendnote}  
\fi

%\renewcommand\enoteformat{\noindent
%\setlength\parindent{12pt}\makebox[0pt][r]{\hyperlink{Hendnotepage.\theenmark}{%
% \hbox{$^{\theenmark}$}}\,}}
%\renewcommand\enotesize{\scriptsize}

%\renewcommand\enoteformat{\noindent
%\leftskip=-2.5em \makebox[2.5em][r]{\theenmark.\ }\hangindent 2.5em}


%-----------------------------------------------------
% Graphics
%-----------------------------------------------------
\usepackage{graphicx}
\ifpdf
  \graphicspath{{pdf/}}
  \pdfcompresslevel=9
  \DeclareGraphicsExtensions{.pdf}
  \DeclareGraphicsRule{.pdf}{pdf}{.pdf}{}
\else
  \graphicspath{{eps/}}
  \DeclareGraphicsExtensions{.eps}
  \DeclareGraphicsRule{.eps}{eps}{.eps}{}
\fi

%\ifx\pdfoutput\undefined
%\usepackage[dvips]{graphicx}
% \graphicspath{{eps/}}
% \DeclareGraphicsExtensions{.eps}
% \DeclareGraphicsRule{.eps}{eps}{.eps}{}
% \else
% \usepackage[pdftex]{graphicx} 
% \graphicspath{{pdf/}}
% \pdfcompresslevel=9
% \DeclareGraphicsExtensions{.pdf}
% \DeclareGraphicsRule{.pdf}{pdf}{.pdf}{}
% \fi

\usepackage{color}
\usepackage{array}

%-----------------------------------------------------
% Makros
%-----------------------------------------------------
%=====================================================
% Makros
%=====================================================

%-----------------------------------------------------
% Simple Substitutions:
%-----------------------------------------------------

\def\ni{\noindent}
\def\usw{$[\dots]$}
\def\daher{$\rightarrow$}
\def\tab{\hspace{2 cm}}
\def\fn{\footnote}
\def\en{\endnote}
\def\Unterschrift{\newline \includegraphics[width=4cm]{unter} \newline}
\def\unterschrift{\Unterschrift}
\newcommand{\bs}{$\backslash$}
 \def\LRA{\Leftrightarrow\mkern40mu}
 \def\RA{\Rightarrow\mkern40mu}
 
%-----------------------------------------------------
% dangerous / ddangerous etc. environments a la Knuth: 
%-----------------------------------------------------

\font\manual=manfnt
\def\dbend{{\manual\char127}}
\def\d@nger{\medbreak\begingroup\clubpenalty=10000
  \def\par{\endgraf\endgroup\medbreak}  
\noindent\hang\hangafter=-2  \hbox 
to0pt{\hskip-\hangindent\dbend\hfill}\ninepoint}
\outer\def\danger{\d@nger}
\def\dd@nger{\medbreak\begingroup\clubpenalty=10000
  \def\par{\endgraf\endgroup\medbreak} 
\noindent\hang\hangafter=-2  \hbox 
to0pt{\hskip-
\hangindent\dbend\kern1pt\dbend\hfill}\ninepoint} 
\outer\def\ddanger{\dd@nger}
\def\enddanger{\endgraf\endgroup}

\newsavebox{\fmbox}
\newenvironment{notdangerous}
{
 \begin{lrbox}{\fmbox}
 \begin{minipage}[t]{1cm}~\end{minipage}
 \begin{minipage}[t]{1.5cm}\hspace{\fill}~\end{minipage}
 \begin{minipage}[t]{0.1cm}~\end{minipage}
 \begin{minipage}[t]{12.5cm}
}
{\end{minipage}\end{lrbox}\usebox{\fmbox}
}
\newenvironment{dangerous}
{
 \begin{lrbox}{\fmbox}
 \begin{minipage}[t]{0.2cm}~\end{minipage}
 \begin{minipage}[t]{1.5cm}\hspace{\fill}\dbend\end{minipage}
 \begin{minipage}[t]{0.1cm}~\end{minipage}
 \begin{minipage}[t]{10.8cm}
}
{\end{minipage}\end{lrbox}\usebox{\fmbox}
}
\newenvironment{ddangerous}
{
 \begin{lrbox}{\fmbox}
 \begin{minipage}[t]{0.2cm}~\end{minipage}
 \begin{minipage}[t]{1.5cm}\hspace{\fill}\dbend\dbend\end{minipage}
 \begin{minipage}[t]{0.1cm}~\end{minipage}
 \begin{minipage}[t]{10.8cm}
}
{\end{minipage}\end{lrbox}\usebox{\fmbox}
}
\newenvironment{dddangerous}
{
 \begin{lrbox}{\fmbox}
 \begin{minipage}[t]{0.2cm}~\end{minipage}
 \begin{minipage}[t]{1.5cm}\hspace{\fill}\dbend\dbend\dbend\end{minipage}
 \begin{minipage}[t]{0.1cm}~\end{minipage}
 \begin{minipage}[t]{10.8cm}
}
{\end{minipage}\end{lrbox}\usebox{\fmbox}
}

%-----------------------------------------------------
% Other environments:
%-----------------------------------------------------

\newcommand{\companyfooter} {
\vspace*{0.2cm}
\setlength{\tabcolsep}{0.05cm}
\tiny
\centerline{
\begin{tabular}{p{1.5cm}p{4.5cm}p{6.0cm}p{0.5cm}p{2.96cm}}
\toprule[0.25pt]
\makebox[1.50cm][l]{Module:} &
\makebox[4.50cm][l]{\mnbook}&
\makebox[6.00cm][c]{\mnname}&
\makebox[0.50cm][l]{$\:$Date:}&
\makebox[2.96cm][r]{\today}
\\%\midrule[0.15pt]
\makebox[1.50cm][l]{Document:} &
\makebox[4.50cm][l]{\mnsubtitle}&
\makebox[6.00cm][c]{\mnsubsubtitle\ (\mnmoduleweek)}&
\makebox[0.50cm][l]{$\:$Version:}&
\makebox[2.96cm][r]{\mnversion}
\\
\bottomrule[0.25pt]
\end {tabular}
}}

