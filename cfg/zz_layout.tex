%=====================================================
% Document Type and Geometry
%=====================================================

\documentclass[12pt,a4paper,fleqn,twoside]{\doctype}	  	
%\write18{if [ ! -f tmp/wc.tex -o ! -s tmp/wc.tex ] ; then make wc verbose=0 loglvl=error fromtex=1 > tmp/wc.tex ; fi}
\input{tmp/env.tex}

%=====================================================
% Some Helpers
%=====================================================

\usepackage{morewrites}								% More write space
\usepackage{lipsum}										% Lorem Impsum...
\usepackage{xparse}										% Parsing of Parameters
\usepackage{xifthen}										% If Then Else
\usepackage{ifpdf}										% Generating PDF or not
\ifpdf\usepackage{cmap}\fi							% Makes PDF searchable, for Skim
\usepackage{xstring}										% String functions
\usepackage[titles]{tocloft}							% New Listofs...

%=====================================================
% Document Geometry
%=====================================================

\usepackage[hmarginratio=1:1]{geometry}
\geometry{
  left=2.5cm,													% left margin
  top=3.5cm,													% top  margin
  textwidth=16cm,											% width of text block
  textheight=21.0cm}										% height of text block
\setlength{\headheight}{1cm}						% height of header
\setlength{\headsep}{1cm}							% distance of header
\setlength{\footskip}{1.5cm}						% distance of footer
\renewcommand{\baselinestretch}{1.0}		% 1line distance
\clubpenalty10000											% Don't want clubss
\widowpenalty10000										% Don't want widows
\hbadness 10000												% Reduce underfull hbox warnings
\sloppy															% Allow sloppy layout, reduce hyphenations


%=====================================================
% Language
%=====================================================

\usepackage[american]{babel}						% Internationalization
\usepackage[utf8]{inputenc}							% Allow for utf-8 in input files
%\usepackage[T1]{fontenc}							% Breaks utf-8 in bibliography							
\usepackage{csquotes}  								% Extended quoting
\selectlanguage{american}							% Select English


%=====================================================
% Fixups for HTML
%=====================================================

\ifthenelse{\equal{\doctype}{book}}{
\ifpdf
\else
  \renewcommand\markboth{\null}			% Markboth makes no sense in HTML
\fi
}{}

\ifpdf
	\usepackage[english]{varioref}					% vref and related	
\else
	\usepackage{tex4ht}
   \newcommand\vref{\ref}							% varioref doesn't work in HTML
   \newcommand\textwidth{\linewidth}
\fi


%=====================================================
% Word counting
%=====================================================

\ifpdf
\else
  \def\HCode#1{}											% Allow plain HTML output
\fi
\newcommand\wcounta{\ifpdf\else\HCode{<!-- COUNT -->}\fi}
\newcommand\wcounte{\ifpdf\else\HCode{<!-- /COUNT -->}\fi }


%=====================================================
% Headers
%=====================================================

\usepackage{fancyhdr}									% Fancy Page Headers
\usepackage[Sonny]{fncychap}					% Fancy Chapter Headers
\ChNumVar{\fontsize{60}{62}}					% Set Font Size for Chapter Headers


%=====================================================
% Floats, Tables, Equations, etc.
%=====================================================

\usepackage{boxedminipage}						% Minipage with a box around it
\usepackage{wrapfig}									% Allow for floats wrapped by text
\usepackage[hang]{caption}							% captions with hanging indent
\usepackage{capt-of} 									% caption{figure}[]{}, outside float
\usepackage[section,above] {placeins}			% Prevent floats floating past the section
\usepackage{float}											% Improved control over floats
\usepackage{booktabs}									% Better tables
\usepackage{adjustbox}								% Flexible boxes, color, rotate, etc.
\usepackage{fancybox}									% Boxes, also for equations
\usepackage{array}										% Better tabular and array environments
\usepackage{colortbl}									% Colored Tables
\usepackage{color}										% Color functionality
\usepackage[table]{xcolor}							% Extended color functionality (blue!20 etc.)


%
% Alter some LaTeX defaults for better treatment of figures:
% See p.105 of "TeX Unbound" for suggested values.
% See pp. 199-200 of Lamport's "LaTeX" book for details.
%  
% General parameters, for ALL pages:
%
\renewcommand{\topfraction}{0.9}			% max fraction of floats at top
\renewcommand{\bottomfraction}{0.8}		% max fraction of floats at bottom
%
% Parameters for TEXT pages (not float pages):
%
\setcounter{topnumber}{2}							% max floats at top of page
\setcounter{bottomnumber}{2}					% max floats at bottom of page
\setcounter{totalnumber}{4}     					% max floats on page
\setcounter{dbltopnumber}{2}    					% for 2-column pages
\renewcommand{\dbltopfraction}{0.9}		% fit big float above 2-col. text
\renewcommand{\textfraction}{0.07}		% allow minimal text w. figs
%
% Parameters for FLOAT pages (not text pages):
%
\renewcommand{\floatpagefraction}{0.7}	% require fuller float pages
% N.B.: floatpagefraction MUST be less than topfraction !!
\renewcommand{\dblfloatpagefraction}{0.7}% require fuller float pages
% Remember to use [htp] or [htpb] for placement


%=====================================================
% Code Listings
%=====================================================

\usepackage{listings}
\definecolor{lstemph}{rgb}{0,0.39,0} 
\definecolor{lstnumbers}{rgb}{0.59,0.57,0.43} 
\definecolor{lstcomments}{rgb}{0.33,0.35,0.69} 
\lstloadlanguages{Java,C++}
\lstset{language=Java,
		extendedchars=true,
        basicstyle=\ttfamily\tiny,
        keywordstyle=\color{lstnumbers},
        identifierstyle=\color{black}, 
        commentstyle=\color{lstcomments},
        stringstyle=\ttfamily\color{blue},
        showstringspaces=true,
        numbers=left,
        stepnumber=1,
        numberstyle=\tiny\ttfamily\color{lstnumbers},
        numbersep=12pt,
        frame=single,
        fontadjust=true,
        xleftmargin=3.5pt,
        xrightmargin=3.5pt,
        escapeinside={(*}{*)}}


%=====================================================
% Hyperref
%=====================================================

\ifpdf
\usepackage[%pdftex,
            pdfpagemode={UseOutlines},
            pdfstartview={FitH},
            colorlinks=true,   
            linkcolor={blue},
            citecolor={blue}, 
            urlcolor={blue},
            bookmarks=true,
            bookmarksopen=true,
            %pdfpagemode=FullScreen,
            %hyperindex=false,
            plainpages=false,
            %hypertexnames=false,
            pdfpagelabels]{hyperref} 
\else
\usepackage[tex4ht]{hyperref}
\fi


%=====================================================
% Massively ugly workaround of hyperrefs issues with equations in glossaries
%=====================================================

%===<exclude> for word count

\ifpdf
  \makeatletter
  \renewcommand{\theHequation}{\@currentHref.\arabic{equation}}
  \gdef\equationgrouping{}
  \makeatother
\fi

%===</exclude>


%=====================================================
% Glossaries
%=====================================================

\usepackage[
nonumberlist, 													% do not show page numbers
acronym,        													% generate acronym listing
toc,                													% show listings as entries in table of contents
section]         													% use section level for toc entries
{glossaries}

%
% Patch Glossaries so that only the first occurrence of a given glossary
% entry is converted to a hyperlink, in order to avoid cluttering.
%
\makeatletter
%% patch first occurence of "\@gls@link[#1]{#2}{\@glo@text}", 
%% as this is the one for \glsused{#2}
\patchcmd{\@gls@}
  {\@gls@link[#1]{#2}{\@glo@text}}
  {\@gls@link[#1,hyper=false]{#2}{\@glo@text}}
  {}{}
\patchcmd{\@glspl@}
  {\@gls@link[#1]{#2}}
  {\@gls@link[#1,hyper=false]{#2}}
  {}{}
\patchcmd{\@Gls@}
  {\@gls@link[#1]{#2}}
  {\@gls@link[#1,hyper=false]{#2}}
  {}{}
\patchcmd{\@Glspl@}
  {\@gls@link[#1]{#2}}
  {\@gls@link[#1,hyper=false]{#2}}
  {}{}
  \patchcmd{\@GLS@}
  {\@gls@link[#1]{#2}{\MakeUppercase{\@glo@text}}}
  {\@gls@link[#1,hyper=false]{#2}{\MakeUppercase{\@glo@text}}}
  {}{}  
\makeatother

%
% Generate a list of symbols
%
\newglossary[slg]{symbolslist}{syi}{syg}{List of Symbols}

%
% Make sure first character of glossary entry name is uppercase
%
\renewcommand{\glsnamefont}[1]{\makefirstuc{#1}}

%
% Remove the dot at the end of glossary descriptions
%
\renewcommand*{\glspostdescription}{}

%
% Activate glossary commands
%
\makeglossaries

%
% Load the glossary definitions the user writes
%
%=====================================================
% Acronyms
%=====================================================

%-----------------------------------------------------
% Samples
%-----------------------------------------------------

% Usage:
% \gls{AD} is pretty interesting. If we do reference a glossary entry,
% like, \gls{glos:AD}, that one of course has to be defined over there.

% An acronym with a glossary entry not hyperlinked
% \newacronym{AD}{AD}{Active Directory\protect\glsadd{glos:AD}}

% Geek Version, with hyperlink to glossary
% \newglossaryentry{AD}{
%   type=\acronymtype,
%   name=AD,
%   first=Active Directory (AD),
%   firstplural={Active Directories (AD's)},
%   description=\glslink{glos:AD}{Active Directory}
% }

%----------------------------------------------------- 
% Content
%-----------------------------------------------------

%<content>%

\newglossaryentry{sox}{
   type=\acronymtype,
   name=SOX,
   first=Sarbanes--Oxley Act (SOX),
   description=\glslink{glos:sox}{Sarbanes--Oxley Act}
 }

\newglossaryentry{sme}{
   type=\acronymtype,
   name=SME,
   first=Small and Medium Enterprise (SME),
   firstplural=Small and Medium Enterprises (SMEs),
   description=\glslink{glos:sme}{Small and Medium Enterprise}
 }

\newglossaryentry{ifrs}{
   type=\acronymtype,
   name=IFRS,
   first=International Financial Reporting Standard (IFRS),
   firstplural=International Financial Reporting Standards (IFRS),
   description=\glslink{glos:ifrs}{International Financial Reporting Standard}
 }
 
 \newglossaryentry{gaap}{
   type=\acronymtype,
   name=GAAP,
   first=Generally accepted accounting principles (GAAP),
   firstplural=Generally accepted accounting principles (GAAP),
   description=\glslink{glos:gaap}{Generally accepted accounting principles}
 }
 
 \newglossaryentry{xbrl}{
   type=\acronymtype,
   name=XBRL,
   first=eXtensible Business Reporting Language (XBRL),
   firstplural=eXtensible Business Reporting Language (XBRL),
   description=\glslink{glos:xbrl}{eXtensible Business Reporting Language}
 }
 
 
 \newglossaryentry{rosf}{
   type=\acronymtype,
   name=ROSF,
   first=Return on Ordinary Shareholder's Funds (ROSF),
   firstplural=Returns on Ordinary Shareholder's Funds (ROSF),
   description=\glslink{glos:rosf}{Returns on Ordinary Shareholder's Funds}
 }
 
 \newglossaryentry{roce}{
   type=\acronymtype,
   name=ROCE,
   first=Return on Capital Employed (ROCE),
   firstplural=Returns on Capital Employed (ROCE),
   description=\glslink{glos:roce}{Return on Capital Employed}
 }
 
 \newglossaryentry{cogs}{
   type=\acronymtype,
   name=COGS,
   first=Cost of Goods Sold (COGS),
   description=\glslink{glos:cogs}{Cost of Goods Sold}
 }
 
 \newglossaryentry{sga}{
   type=\acronymtype,
   name=SG\&A,
   first={Selling, General and Administrative Expenses (SG\&A)},
   description=\glslink{glos:sga}{Selling, General and Administrative Expenses}
 }

 \newglossaryentry{roa}{
   type=\acronymtype,
   name=ROA,
   first=Return on Assets (ROA),
   description=\glslink{glos:roa}{Return on Assets (ROA)}
 }
 
 
 
 
 
 

%</content>%
  
  
























%=====================================================
% Symbols
%=====================================================

%-----------------------------------------------------
% Samples
%-----------------------------------------------------

% Usage:
% \section{Some Greek symbols}
% If you calculate with \gls{symb:Pi} you always get an irrational result, because 
% \gls{symb:Pi} itself is irrational. As a matter of fact, there are \gls{symb:Phi} 
% and \gls{symb:Lambda}, too.

%Some entries for the list of symbols
\newglossaryentry{symb:Pi}{
  type=symbolslist,
  name=$\pi$,
  description={A mathematical constant whose value is the ratio of any circle's circumference to its diameter.},
  sort=symbolpi
}
\newglossaryentry{symb:Phi}{
  type=symbolslist,
  name=$\varphi$,
  description={An angle.},
  sort=symbolphi
}
\newglossaryentry{symb:Lambda}{
  type=symbolslist,
  name=$\lambda$,
  description={Lambda indictes an eigenvalue in the mathematics  of linear algebra.},
  sort=symbollambda
}

%-----------------------------------------------------
% Content
%-----------------------------------------------------

%<content>%



%</content>%


%-----------------------------------------------------
% Default Content
%-----------------------------------------------------


%-----------------------------------------------------
% AP
%-----------------------------------------------------

\def\glosdescap{Also called \emph{Actual Rate}. This is a 
calculated value that describes the price, or rate, of a given
cost factor, such as nursing, on an per-hour basis. As an \emph{actual}
value, it is calculated as follows:
 
\begin{equation}
   AP = \frac{\text{Actual Costs}}{\text{Total Actual Hours}}
\end{equation} 
}
\newglossaryentry{glos:ap}{
  name=Actual Price,
  description={\glosdescap}
}
\def\symbdescap{See \glslink{glos:ap}{Actual Price}.}
\newglossaryentry{symb:ap}{
  type=symbolslist,
  name=AP,
  description={\symbdescap},
  sort=symbolap
}

%-----------------------------------------------------
% AQ
%-----------------------------------------------------
\def\glosdescaq{This is a 
measured value that describes the \emph{actual} quantity
of a product in monetary terms; for observations of working
hours, it is typically called  \emph{Actual Time}.
}
\newglossaryentry{glos:aq}{
  name=Actual Quantity,
  description={\glosdescaq}
}
\def\symbdescaq{See \glslink{glos:aq}{Actual Quantity}.}
\newglossaryentry{symb:aq}{
  type=symbolslist,
  name=AQ,
  description={\symbdescaq},
  sort=symbolaq
}


%-----------------------------------------------------
% SP
%-----------------------------------------------------

\def\glosdescsp{Also called \emph{Standard Rate}. This is a 
calculated value that describes the price, or rate, of a given
cost factor, such as nursing, on an per-hour basis. As a \emph{standard},
or \emph{budgeted} value, it is calculated as follows:
 
\begin{equation}
   SP = \frac{\text{Budgeted Costs}}{\text{Total Budgeted Hours}}
\end{equation} 
}
\newglossaryentry{glos:sp}{
  name=Standard Price,
  description={\glosdescsp}
}
\def\symbdescsp{See \glslink{glos:sp}{Standard Price}.}
\newglossaryentry{symb:sp}{
  type=symbolslist,
  name=SP,
  description={\symbdescsp},
  sort=symbolsp
}



%-----------------------------------------------------
% SQ
%-----------------------------------------------------
\def\glosdescsq{This is a 
measured value that describes the \emph{budgeted} quantity
of a product in monetary terms; for observations of working
hours, it is typically called  \emph{Standard Time}.
If the \glslink{glos:ao}{actual output} differs
significantly from the \glslink{glos:so}{budgeted output},
it is necessary to utilize \glslink{glos:rsq}{Revised Standard Quantity}
instead.
}
\newglossaryentry{glos:sq}{
  name=Standard Quantity,
  description={\glosdescsq}
}
\def\symbdescsq{See \glslink{glos:sq}{Standard Quantity}.}
\newglossaryentry{symb:sq}{
  type=symbolslist,
  name=SQ,
  description={\symbdescsq},
  sort=symbolsq
}

%-----------------------------------------------------
% RSQ
%-----------------------------------------------------
\def\glosdescrsq{This is a calculated value that
describes the \emph{budgeted} quantity of a  
product in monetary terms, corrected by the
ratio of \glslink{glos:ao}{actual output} to
\glslink{glos:so}{budgeted output}. For observations
of working hours, it is typically called \emph{Revised Standard Time}.
It is calculated as follows:
 
\begin{equation}
  RSQ = \glslink{symb:sq}{SQ} \: \frac{\glslink{symb:ao}{AO}}{\glslink{symb:so}{SO}}
\end{equation} 

Since the equation resolves to the value of \glslink{symb:sq}{SQ} if
\glslink{symb:ao}{AO} and \glslink{symb:so}{SO} are identical,
RSQ  should always be used instead of \glslink{symb:sq}{SQ}.
}
\newglossaryentry{glos:rsq}{
  name=Revised Standard Quantity,
  description={\glosdescrsq}
}
\def\symbdescrsq{See \glslink{glos:rsq}{Revised Standard Quantity}.}
\newglossaryentry{symb:rsq}{
  type=symbolslist,
  name=RSQ,
  description={\symbdescrsq},
  sort=symbolrsq
}

%-----------------------------------------------------
% SO
%-----------------------------------------------------
\def\glosdescso{Also called \emph{Standard Yield}. This is a 
measured value that describes the \emph{budgeted} output,
in terms of quantity, of a given activity. 
}
\newglossaryentry{glos:so}{
  name=Standard Output,
  description={\glosdescso}
}
\def\symbdescso{See \glslink{glos:so}{Standard Output}.}
\newglossaryentry{symb:so}{
  type=symbolslist,
  name=SO,
  description={\symbdescso},
  sort=symbolso
}

%-----------------------------------------------------
% AO
%-----------------------------------------------------
\def\glosdescao{Also called \emph{Actual Yield}. This is a 
measured value that describes the \emph{actual} output,
in terms of quantity, of a given activity.  
}
\newglossaryentry{glos:ao}{
  name=Actual Output,
  description={\glosdescao}
}
\def\symbdescao{See \glslink{glos:ao}{Actual Output}.}
\newglossaryentry{symb:ao}{
  type=symbolslist,
  name=AO,
  description={\symbdescao},
  sort=symbolao
}

%-----------------------------------------------------
% TPU
%-----------------------------------------------------
\def\glosdesctpu{Also called \emph{Productivity}. This is a 
\emph{measured} value that describes the time required to
produce one unit of output.
}
\newglossaryentry{glos:tpu}{
  name=Time per Unit,
  description={\glosdesctpu}
}
\def\symbdesctpu{See \glslink{glos:tpu}{Time per Unit}.}
\newglossaryentry{symb:tpu}{
  type=symbolslist,
  name=TPU,
  description={\symbdesctpu},
  sort=symboltpu
}

%-----------------------------------------------------
% OCV
%-----------------------------------------------------
\def\glosdescocv{This is a 
calculated value describes the difference between
\emph{actual}  expenses and \emph{budgeted}
expenses in monetary terms.

It is calculated as follows:\footnote{The calculation was
derived from \cite{Averkamp:2012locv}.}
 
\begin{equation}
   OCV = \glslink{symb:rsq}{RSQ} \times \glslink{symb:sp}{SP} -
                              \glslink{symb:aq}{AQ} \times \glslink{symb:sp}{AP}
\end{equation} 

\begin{dangerous}
A negative value is called \emph{unfavorable}. It is considered to be in the responsibility of
\emph{departmental management}  as far as they are able to exert control over the costs contributing to
the variance.
\end{dangerous}
}
\newglossaryentry{glos:ocv}{
  name=Overhead Controllable Variance,
  description={\glosdescocv}
}
\def\symbdescocv{See \glslink{glos:ocv}{Overhead Controllable Variance}.}
\newglossaryentry{symb:ocv}{
  type=symbolslist,
  name=OCV,
  description={\symbdescocv},
  sort=symbolocv
}

%-----------------------------------------------------
% OVV
%-----------------------------------------------------
\def\glosdescovv{This is a 
calculated value that describes the difference between the \emph{budget}  allowed
and the \emph{budgeted}  expenses to the \emph{actual} work in progress.

It is calculated as follows:\footnote{The calculation was
derived from \cite{Averkamp:2012ovv}.}
 
\begin{equation}
   OVV = \glslink{symb:sp}{SP}_{fixed\:costs} \times \left( \glslink{symb:ao}{AO} \times
   \glslink{symb:tpu}{Productivity} - \glslink{symb:rsq}{RSQ}   \right)
\end{equation}

\begin{dangerous}
A negative value is called \emph{unfavorable}. In that case, the value expresses the cost of 
capacity available but not utilized efficiently. It is considered to be in the responsibility of
\emph{executive management}  and \emph{departmental management}.
\end{dangerous}
}
\newglossaryentry{glos:ovv}{
  name=Overhead Volume Variance,
  description={\glosdescovv}
}
\def\symbdescovv{See \glslink{glos:ovv}{Overhead Volume Variance}.}
\newglossaryentry{symb:ovv}{
  type=symbolslist,
  name=OVV,
  description={\symbdescovv},
  sort=symbolovv
}

%-----------------------------------------------------
% OSV
%-----------------------------------------------------
\def\glosdescosv{This is a 
calculated value that describes the difference between the \emph{actual}
expenses expenses and the \emph{budget} allowed based on the \emph{actual}
quantity produced or hours worked, respectively.   

It is calculated as follows:\footnote{The calculation was
derived from \cite{Averkamp:2012osv}.}
 
\begin{equation}
  OSV = \glslink{symb:aq}{AQ} \times \glslink{symb:ap}{AP} -
              \glslink{symb:aq}{AQ} \times \glslink{symb:ap}{AP}_{overhead} -
              \text{Actual Variable Costs}
\end{equation}

\begin{dangerous}
A negative value is called \emph{unfavorable}. It is considered to be in the area of
responsibility of the \emph{department manager} who has to keep actual expenses
within budgeted limits.
\end{dangerous}
}
\newglossaryentry{glos:osv}{
  name=Overhead Spending Variance,
  description={\glosdescosv}
}
\def\symbdescosv{See \glslink{glos:osv}{Overhead Spending Variance}.}
\newglossaryentry{symb:osv}{
  type=symbolslist,
  name=OSV,
  description={\symbdescosv},
  sort=symbolosv
}


%-----------------------------------------------------
% OEV
%-----------------------------------------------------
\def\glosdescoev{This is a 
calculated value that describes the difference between the \emph{actual}
expenses expenses and the \emph{budget} allowed based on the \emph{actual}
quantity produced or hours worked, respectively.   

It is calculated as follows:\footnote{The calculation was
derived from \cite{Averkamp:2012oev}.}
 
\begin{equation}
  OEV = \left( \glslink{symb:ap}{AP}_{overhead} + \glslink{symb:sp}{SP}_{fixed\:costs} \right) \times
              \left(  \glslink{symb:aq}{AQ} - \glslink{symb:ao}{AO} \times
              \glslink{symb:tpu}{Productivity}  \right)
\end{equation}

\begin{dangerous}
A negative value is called \emph{unfavorable} and typically caused by inefficiencies
such as inexperienced labor, changes in operations, introduction of new procedures, 
work tools or materials. It is considered to be in the area of
responsibility of the \emph{department manager}.
\end{dangerous}
}
\newglossaryentry{glos:oev}{
  name=Overhead Efficiency Variance,
  description={\glosdescoev}
}
\def\symbdescoev{See \glslink{glos:oev}{Overhead Efficiency Variance}.}
\newglossaryentry{symb:oev}{
  type=symbolslist,
  name=OEV,
  description={\symbdescoev},
  sort=symboloev
}

%-----------------------------------------------------
% OIV
%-----------------------------------------------------
\def\glosdescoiv{This is a 
calculated value that describes the difference between the
budget allowed based on \emph{actual} hours worked, multiplied
with the \emph{budgeted} overhead rate. 

As an \emph{actual}
value, it is calculated as follows:\footnote{The calculation was
derived from \cite{Averkamp:2012oiv}.}
 
\begin{equation}
  OIV = \left( \glslink{symb:aq}{AQ} - \glslink{symb:rsq}{RSQ} \right) \times
             \glslink{symb:sp}{SP}_{fixed\:costs}
\end{equation}

\begin{dangerous}
A negative value is called \emph{unfavorable} and indicates the 
amount of overhead that is under absorbed due to \emph{actual}
hours being lower than \emph{budgeted} hours on which the
calculation of the overhead rate was based. It is considered to be in the area of
responsibility of the \emph{department manager}.
\end{dangerous}
}
\newglossaryentry{glos:oiv}{
  name=Overhead Idle Capacity Variance,
  description={\glosdescoiv}
}
\def\symbdescoiv{See \glslink{glos:oiv}{Overhead Idle Capacity Variance}.}
\newglossaryentry{symb:oiv}{
  type=symbolslist,
  name=OIV,
  description={\symbdescoiv},
  sort=symboloiv
}


%-----------------------------------------------------
% OOV
%-----------------------------------------------------
\def\glosdescoov{Also called \emph{Net Factory Overhead Variance}. 
An overall figure, it is considered to be in the responsibility of
\emph{executive management} and is a 
calculated value that describes the difference between the \emph{actual} 
overhead and expenses incurred using the \emph{budgeted} overhead rate. 

It is calculated as follows:\footnote{The calculation was
derived from \cite{Averkamp:2012oov}.}
 
\begin{equation}
  OOV = \left( \glslink{symb:ap}{AP}_{overhead} + \glslink{symb:sp}{SP}_{fixed\:costs} \right) \times
              \glslink{symb:ao}{AO} \times \glslink{symb:tpu}{Productivity} -
              \glslink{symb:aq}{AQ} \times \glslink{symb:ap}{AP}
\end{equation}

\begin{dangerous}
A negative value is called \emph{unfavorable} and typically requires further
analysis to guide management to solve the situation. In practice, this requires
to look at the other overhead variances, such as \glslink{symb:ocv}{OCV},
 \glslink{symb:ovv}{OVV} (two-variances method), but also  \glslink{symb:osv}{OSV},
 \glslink{symb:oev}{OEV} and \glslink{symb:oiv}{OIV} (three-variances method)
 to understand better the causes of the issues; using the \glslink{symb:osv}{OSV}
 and  variable \glslink{symb:oev}{OEV} on one side and the fixed  \glslink{symb:oev}{OEV}
 and   \glslink{symb:oiv}{OIV} (four-variances method), one splits the 
  \glslink{symb:oev}{OEV}  into its fixed and variable components.
\end{dangerous}
}
\newglossaryentry{glos:oov}{
  name=Overall Overhead Variance,
  description={\glosdescoov}
}
\def\symbdescoov{See \glslink{glos:oov}{Overall Overhead Variance}.}
\newglossaryentry{symb:oov}{
  type=symbolslist,
  name=OOV,
  description={\symbdescoov},
  sort=symboloov
}

%-----------------------------------------------------
% LCV
%-----------------------------------------------------
\def\glosdesclcv{This  is a 
calculated value that describes the difference between the \emph{budgeted} cost
for production and the \emph{actual} cost for production.  

Comparing Labor costs with Material costs,
LCV corresponds to \glslink{symb:mcv}{MCV}.

It is calculated as follows:\footnote{The calculation was
derived from \cite{Poudel:2012lcv}.}
 
\begin{equation}
  LCV = \left( \glslink{symb:rsq}{RSQ} \times \glslink{symb:sp}{SP}\right) - 
              \left( \glslink{symb:aq}{AQ} \times \glslink{symb:ap}{AP}\right)
\end{equation}

\begin{dangerous}
A negative value is called \emph{unfavorable} and typically means that
more hours are utilized than are allowed budgeted for, pointing to
possibly inefficient use of labor time due to lack of automation
or otherwise inefficient production methods (\glslink{glos:lev}{LEV})
and possibly increased hourly charges \emph{actual} vs. \emph{budgeted}
(\glslink{glos:lrv}{LRV}).   
\end{dangerous}
}
\newglossaryentry{glos:lcv}{
  name=Labor Cost Variance,
  description={\glosdesclcv}
}
\def\symbdesclcv{See \glslink{glos:lcv}{Labor Cost Variance}.}
\newglossaryentry{symb:lcv}{
  type=symbolslist,
  name=LCV,
  description={\symbdesclcv},
  sort=symbollcv
}

%-----------------------------------------------------
% MCV
%-----------------------------------------------------
\def\glosdescmcv{This  is a 
calculated value that describes the difference between the \emph{budgeted} cost
for production and the \emph{actual} cost for production.  

Comparing Labor costs with Material costs,
MCV corresponds to \glslink{symb:lcv}{LCV}.

It is calculated as follows:\footnote{The calculation was
derived from \cite{Poudel:2011mcv}.}
 
\begin{equation}
  MCV = \left( \glslink{symb:rsq}{RSQ} \times \glslink{symb:sp}{SP}\right) - 
              \left( \glslink{symb:aq}{AQ} \times \glslink{symb:ap}{AP}\right)
\end{equation}

\begin{dangerous}
A negative value is called \emph{unfavorable} and typically means that
more material is utilized than are allowed budgeted for, pointing to
possibly inefficient use of material due to inefficient production 
methods (\glslink{glos:muv}{MUV})
and possibly increased material costs \emph{actual} vs. \emph{budgeted}
(\glslink{glos:mpv}{MPV}).   
\end{dangerous}
}
\newglossaryentry{glos:mcv}{
  name=Material Cost Variance,
  description={\glosdescmcv}
}
\def\symbdescmcv{See \glslink{glos:mcv}{Material Cost Variance}.}
\newglossaryentry{symb:mcv}{
  type=symbolslist,
  name=MCV,
  description={\symbdescmcv},
  sort=symbolmcv
}

%-----------------------------------------------------
% LRV
%-----------------------------------------------------
\def\glosdesclrv{This  is a 
calculated value that describes the difference between the \emph{budgeted} cost
and the \emph{actual} cost paid for the \emph{actual} number of hours.

Comparing Labor costs with Material costs,
LRV corresponds to \glslink{symb:mpv}{MPV}.

It is calculated as follows:\footnote{The calculation was
derived from \cite{Poudel:2012lrv}.}
 
\begin{equation}
  LRV =  \glslink{symb:aq}{AQ} \times 
              \left( \glslink{symb:sp}{SP} - \glslink{symb:ap}{AP}\right)
\end{equation}

\begin{dangerous}
A negative value is called \emph{unfavorable} and typically points to
increased \emph{actual} hourly rates compared to what was
\emph{budgeted}.
\end{dangerous}
}
\newglossaryentry{glos:lrv}{
  name=Labor Rate Variance,
  description={\glosdesclrv}
}
\def\symbdesclrv{See \glslink{glos:lrv}{Labor Rate Variance}.}
\newglossaryentry{symb:lrv}{
  type=symbolslist,
  name=LRV,
  description={\symbdesclrv},
  sort=symbollrv
}

%-----------------------------------------------------
% MPV
%-----------------------------------------------------
\def\glosdescmpv{This  is a 
calculated value that describes the difference between the \emph{budgeted} cost
and the \emph{actual} cost paid for the \emph{actual} material utilized.

Comparing Labor costs with Material costs,
MPV corresponds to \glslink{symb:lrv}{LRV}.

It is calculated as follows:\footnote{The calculation was
derived from \cite{Poudel:2011mpv}.}
 
\begin{equation}
  MPV =  \glslink{symb:aq}{AQ} \times 
              \left( \glslink{symb:sp}{SP} - \glslink{symb:ap}{AP}\right)
\end{equation}

\begin{dangerous}
A negative value is called \emph{unfavorable} and typically points to
increased \emph{actual} price for material compared to what was
\emph{budgeted}.
\end{dangerous}
}
\newglossaryentry{glos:mpv}{
  name=Material Price Variance,
  description={\glosdescmpv}
}
\def\symbdescmpv{See \glslink{glos:mpv}{Material Price Variance}.}
\newglossaryentry{symb:mpv}{
  type=symbolslist,
  name=MPV,
  description={\symbdescmpv},
  sort=symbolmpv
}

%-----------------------------------------------------
% LEV
%-----------------------------------------------------
\def\glosdesclev{This  is a 
calculated value that compares the \emph{actual} number of hours it took to
create an actual output with the number of hours \emph{budgeted}
for that output.      

Comparing Labor costs with Material costs,
LEV corresponds to \glslink{symb:muv}{MUV}.

It is calculated as follows:\footnote{The calculation was
derived from \cite{Poudel:2011lev}.}
 
\begin{equation}
  LEV =  \glslink{symb:aq}{AQ} \times 
              \left( \glslink{symb:sp}{SP} - \glslink{symb:ap}{AP}\right)
\end{equation}

\begin{dangerous}
A negative value is called \emph{unfavorable} and typically means that
more hours are utilized than are allowed budgeted for, pointing to
possibly inefficient use of labor time due to lack of automation
or otherwise inefficient production methods.
\end{dangerous}
}
\newglossaryentry{glos:lev}{
  name=Labor Efficiency Variance,
  description={\glosdesclev}
}
\def\symbdesclev{See \glslink{glos:lev}{Labor Efficiency Variance}.}
\newglossaryentry{symb:lev}{
  type=symbolslist,
  name=LEV,
  description={\symbdesclev},
  sort=symbollev
}

%-----------------------------------------------------
% MUV
%-----------------------------------------------------
\def\glosdescmuv{This  is a 
calculated value that compares the \emph{actual} amount of material it took to
create an actual output with the amoutn of material \emph{budgeted}
for that output.      

Comparing Labor costs with Material costs,
MUV corresponds to \glslink{symb:lev}{LEV}.

It is calculated as follows:\footnote{The calculation was
derived from \cite{Poudel:2011muv}.}
 
\begin{equation}
  MUV =  \glslink{symb:aq}{AQ} \times 
              \left( \glslink{symb:sp}{SP} - \glslink{symb:ap}{AP}\right)
\end{equation}

\begin{dangerous}
A negative value is called \emph{unfavorable} and typically means that
more material is utilized than are allowed budgeted for, pointing to
possibly inefficient use of material due to inefficient production methods.
\end{dangerous}
}
\newglossaryentry{glos:muv}{
  name=Material Usage Variance,
  description={\glosdescmuv}
}
\def\symbdescmuv{See \glslink{glos:muv}{Material Usage Variance}.}
\newglossaryentry{symb:muv}{
  type=symbolslist,
  name=MUV,
  description={\symbdescmuv},
  sort=symbolmuv
}

%-----------------------------------------------------
% LYV
%-----------------------------------------------------
\def\glosdesclyv{This  is a 
calculated value that identifies the portion of the
\glslink{glos:lev}{LEV} to obtain a favorable
or unfavorable yield.

Comparing Labor costs with Material costs,
LYV corresponds to \glslink{symb:myv}{MYV}.

It is calculated as follows:\footnote{The calculation was
derived from \cite{Poudel:2011lyv}.}
 
\begin{equation}
  LYV =  \glslink{symb:sp}{SP} \times 
              \left( \glslink{symb:ao}{AO} - \glslink{symb:so}{SO}\right)
\end{equation}

\begin{dangerous}
A negative value is called \emph{unfavorable} and typically 
points to inefficiencies---see \glslink{glos:lev}{LEV}---when
creating a given output.
\end{dangerous}
}
\newglossaryentry{glos:lyv}{
  name=Labor Yield Variance,
  description={\glosdesclyv}
}
\def\symbdesclyv{See \glslink{glos:lyv}{Labor Yield Variance}.}
\newglossaryentry{symb:lyv}{
  type=symbolslist,
  name=LYV,
  description={\symbdesclyv},
  sort=symbollyv
}

%-----------------------------------------------------
% MYV
%-----------------------------------------------------
\def\glosdescmyv{This  is a 
calculated value that identifies the portion of the
\glslink{glos:muv}{MUV} to obtain a favorable
or unfavorable yield.

Comparing Labor costs with Material costs,
MYV corresponds to \glslink{symb:lyv}{LYV}.

It is calculated as follows:\footnote{The calculation was
derived from \cite{Poudel:2011myv}.}
 
\begin{equation}
  MYV =  \glslink{symb:sp}{SP} \times 
              \left( \glslink{symb:ao}{AO} - \glslink{symb:so}{SO}\right)
\end{equation}

\begin{dangerous}
A negative value is called \emph{unfavorable} and typically 
points to inefficiencies---see \glslink{glos:muv}{MUV}---when
creating a given output.
\end{dangerous}
}
\newglossaryentry{glos:myv}{
  name=Material Yield Variance,
  description={\glosdescmyv}
}
\def\symbdescmyv{See \glslink{glos:myv}{Material Yield Variance}.}
\newglossaryentry{symb:myv}{
  type=symbolslist,
  name=MYV,
  description={\symbdescmyv},
  sort=symbolmyv
}











































%=====================================================
% Glossary
%=====================================================

%-----------------------------------------------------
% Samples
%-----------------------------------------------------

% Usage:
% \gls{glos:AD} is pretty interesting. If we have a cross reference from
% the acronyms, we can also directly go to that using \gls{AD}; this
% requires then that over there, we have something like
%  description=\glslink{glos:AD}{Active Directory}

% \newglossaryentry{glos:AD}{
% name=Active Directory,
% description={Active Directory is the directory service for 
% Windows based networks, that allows central organization and 
% administration of any network resource.
% It allows a single-sign-on concept independent from network 
% topologies or network protocols. As a prerequisite you need 
% a Windows Server acting as Domain Controller. This computer 
% stores all necessary data, e.\,g.~usernames and corresponding 
% passwords.}
% }


%-----------------------------------------------------
% Content
%-----------------------------------------------------

%<content>%



%</content>%

































%
% These commands actually create / update the different
% indices / glossaries
%
%makeindex -s document.ist -t document.alg -o document.acr document.acn
%makeindex -s document.ist -t document.glg -o document.gls document.glo
%makeindex -s document.ist -t document.slg -o document.syi document.syg
%makeindex document


%=====================================================
% Index
%=====================================================

\usepackage{makeidx}


%=====================================================
% Graphics
%=====================================================
\usepackage{graphicx}
\ifpdf
  \graphicspath{{pdf/}}
  \pdfcompresslevel=9
  \DeclareGraphicsExtensions{.pdf}
  \DeclareGraphicsRule{.pdf}{pdf}{.pdf}{}
\else
  \graphicspath{{eps/}}
  \DeclareGraphicsExtensions{.eps}
  \DeclareGraphicsRule{.eps}{eps}{.eps}{}
\fi


%=====================================================
% Media
%=====================================================

%\renewcommand{\video}[6]{% file xpos ypos width height controls
%  \vspace{#3}\hspace{#2}{\pdfannot width #4 height #5 depth 0cm {%
%   /Subtype /Movie  
%   /Movie  << /F (#1) >> 
%   /A << /ShowControls #6 /Rate 1 >>
%   }}}
%\fi

\usepackage{sty/easymovie/easymovie}


%=====================================================
% Equations
%=====================================================

\usepackage[fleqn,tbtags]{mathtools}			% Mathematical Processing
\usepackage{amssymb}									% Scientific Symbols
\usepackage{latexsym}									%	Scientific Symbols
\mathtoolsset{showonlyrefs}						% Label only (eqref) referenced Equations 
%\everymath{\rm}											% Default to roman style

\usepackage[customcolors]{hf-tikz}				% Highlight Formulas
\usetikzlibrary{calc}
\tikzstyle{every picture}+=[remember picture]
\hfsetfillcolor{blue!10}
\hfsetbordercolor{blue}


%=====================================================
% Font Settings
%=====================================================

\usepackage{microtype}								% More precise typography
\usepackage{fix-cm}										% Permit arbitrary font sizes
\usepackage{eurosym}									% Euro Symbol

%
% Specify fonts for Captions, Sectionts, etc.
%
\renewcommand*\captionlabelfont{\bfseries}	
\renewcommand*\captionsize{\itshape}		

\makeatletter
\renewcommand{\section}{\@startsection{section}{1}{\z@}%
    {-2.2ex \@plus-1ex \@minus -.2ex}{1.3ex \@plus.2ex}%
    {\reset@font\large\bfseries}}
\renewcommand{\subsection}{\@startsection{subsection}{2}{\z@}%
    {-1.5ex \@plus -1ex \@minus-.2ex}{0.8ex \@plus.2ex}%
    {\reset@font\normalsize\bfseries}}
\renewcommand{\subsubsection}{\@startsection{subsubsection}{3}{\z@}%
     {-1.2ex\@plus -1ex \@minus -.2ex}{0.5ex \@plus .2ex}%
     {\reset@font\normalsize}}
 \renewcommand{\paragraph}{\@startsection{paragraph}{4}{0mm}%
  {1ex \@plus1ex \@minus.2ex}%
  {-1em}%
  {\normalfont\normalsize\it}}
 \renewcommand{\subparagraph}{\@startsection{subparagraph}{5}{\parindent}%
  {2.0ex \@plus1ex \@minus .2ex}%
  {-1em}%
  {\normalfont\normalsize\it}}     
\makeatother

%
% Itemizes
%
\renewcommand{\labelitemi}{$\triangleright$}
\renewcommand*\descriptionlabel[1]{\hspace\labelsep
                                \normalfont\itshape #1}

%
% Save Default Space above Itemize etc., then set it to 0
%
% Not resetting, need further investigation.
%
% \newlength{\oldabovedisplayskip}
% \setlength{\oldabovedisplayskip}{\abovedisplayskip}
% %\setlength{\abovedisplayskip}{0pt}
% \expandafter\def\expandafter\normalsize\expandafter{%
% \normalsize\setlength\oldabovedisplayskip{\abovedisplayskip}}
% \expandafter\def\expandafter\normalsize\expandafter{%
% \normalsize\setlength\abovedisplayskip{0pt}}
% \expandafter\def\expandafter\normalsize\expandafter{%
% \normalsize\setlength\abovedisplayskip{\oldabovedisplayskip}}


%=====================================================
% Table of... / Listofs
%=====================================================

\usepackage[titles]{tocloft}							% New Listofs...
\setcounter{secnumdepth}{10}						% Section numbers down to level 10
\setcounter{tocdepth}{3}								% TOC content down to level 3

%
% Give some more room for page numbers
%
\makeatletter
\renewcommand{\@pnumwidth}{3em} 
\renewcommand{\@tocrmarg}{4em}
\makeatother


%=====================================================
% Footnotes / Endnotes
%=====================================================

\usepackage[flushmargin, hang]{footmisc}	% More footnote options
\usepackage{sty/botfnote/botfnote}				% Force footnotes to the bottom
\usepackage[backref,counter-format=arabic]{sty/enotez/enotez} % Backreferencing Endnotes
 
\ifpdf
  \usepackage{sty/hyperendnote/hyperendnote} %  Referenced Endnotes
\fi


\usepackage{setspace}

\DeclareInstance{enotez-list}{itemize}{list}{
  list-type = itemize,
  number = \enmark{#1} ,
  format = \footnotesize,
}

%=====================================================
% Load Macros
%=====================================================    

%=====================================================
% Makros
%=====================================================

%-----------------------------------------------------
% Simple Substitutions:
%-----------------------------------------------------

\def\ni{\noindent}
\def\usw{$[\dots]$}
\def\daher{$\rightarrow$}
\def\tab{\hspace{2 cm}}
\def\fn{\footnote}
\def\en{\endnote}
\def\Unterschrift{\newline \includegraphics[width=4cm]{fig/unter} \newline}
\def\unterschrift{\Unterschrift}
\newcommand{\bs}{$\backslash$}

%
% Mathematical proofs
%
\def\LRA{\Leftrightarrow\mkern40mu}
\def\RA{\Rightarrow\mkern40mu}
\newcommand{\qed}{\hfill \ensuremath{\hfill \blacksquare}}

%
% Currency Symbols
%
\def\gbpm{\text{\,M\pounds}}
\def\gbp{\text{\,\pounds}}
\def\usdm{\text{\,M\$}}
\def\usd{\text{\,\$}}
\def\eurm{\text{\,M\euro}}
\def\eur{\text{\,\euro}}
\def\perc{\text{~\%}}
 
 
%-----------------------------------------------------
% dangerous / ddangerous etc. environments a la Knuth: 
%-----------------------------------------------------

\font\manual=manfnt
\def\dbend{{\manual\char127}}
\def\d@nger{\medbreak\begingroup\clubpenalty=10000
  \def\par{\endgraf\endgroup\medbreak}  
\noindent\hang\hangafter=-2  \hbox 
to0pt{\hskip-\hangindent\dbend\hfill}\ninepoint}
\outer\def\danger{\d@nger}
\def\dd@nger{\medbreak\begingroup\clubpenalty=10000
  \def\par{\endgraf\endgroup\medbreak} 
\noindent\hang\hangafter=-2  \hbox 
to0pt{\hskip-
\hangindent\dbend\kern1pt\dbend\hfill}\ninepoint} 
\outer\def\ddanger{\dd@nger}
\def\enddanger{\endgraf\endgroup}

\newsavebox{\fmbox}
\newenvironment{notdangerous}
{
 \begin{lrbox}{\fmbox}
 \begin{minipage}[t]{1cm}~\end{minipage}
 \begin{minipage}[t]{1.5cm}\hspace{\fill}~\end{minipage}
 \begin{minipage}[t]{0.1cm}~\end{minipage}
 \begin{minipage}[t]{12.5cm}
}
{\end{minipage}\end{lrbox}\usebox{\fmbox}
}
\newenvironment{dangerous}
{
 \begin{lrbox}{\fmbox}
 \begin{minipage}[t]{0.2cm}~\end{minipage}
 \begin{minipage}[t]{1.5cm}\hspace{\fill}\dbend\end{minipage}
 \begin{minipage}[t]{0.1cm}~\end{minipage}
 \begin{minipage}[t]{10.8cm}
}
{\end{minipage}\end{lrbox}\usebox{\fmbox}
}
\newenvironment{ddangerous}
{
 \begin{lrbox}{\fmbox}
 \begin{minipage}[t]{0.2cm}~\end{minipage}
 \begin{minipage}[t]{1.5cm}\hspace{\fill}\dbend\dbend\end{minipage}
 \begin{minipage}[t]{0.1cm}~\end{minipage}
 \begin{minipage}[t]{10.8cm}
}
{\end{minipage}\end{lrbox}\usebox{\fmbox}
}
\newenvironment{dddangerous}
{
 \begin{lrbox}{\fmbox}
 \begin{minipage}[t]{0.2cm}~\end{minipage}
 \begin{minipage}[t]{1.5cm}\hspace{\fill}\dbend\dbend\dbend\end{minipage}
 \begin{minipage}[t]{0.1cm}~\end{minipage}
 \begin{minipage}[t]{10.8cm}
}
{\end{minipage}\end{lrbox}\usebox{\fmbox}
}

%-----------------------------------------------------
% Other environments:
%-----------------------------------------------------

\newcommand{\companyfooter} {
\vspace*{0.2cm}
\setlength{\tabcolsep}{0.05cm}
\tiny
\centerline{
\begin{tabular}{p{1.5cm}p{4.5cm}p{6.0cm}p{0.5cm}p{2.96cm}}
\toprule[0.25pt]
\makebox[1.50cm][l]{Module:} &
\makebox[4.50cm][l]{\mnbook}&
\makebox[6.00cm][c]{\mnname}&
\makebox[0.50cm][l]{$\:$Date:}&
\makebox[2.96cm][r]{\today}
\\%\midrule[0.15pt]
\makebox[1.50cm][l]{Document:} &
\makebox[4.50cm][l]{\mnsubtitle}&
\makebox[6.00cm][c]{\mnsubsubtitle\ (\mnmoduleweek)}&
\makebox[0.50cm][l]{$\:$Version:}&
\makebox[2.96cm][r]{\mnversion}
\\
\bottomrule[0.25pt]
\end {tabular}
}}


%-----------------------------------------------------
% Exhibit:
%
% Use:
%
% \begin{exhibit}[TOCCaption]{Caption}[rRiIlLoO][.5][32]
%
% Parameters:
%
% 1. Optional parameter TOCCaption gives the
%     entry into \listofexhibit
%
% 2. Mandatory parameter: Caption
%
% 3. Optional parameter:  Lines to wrap around.
%     Not needed unless positioned to the right with e.g.
%     an itemize to the left. In this case, specify the number
%     of lines running left of the exhibit.
%
% 4. Optional parameter: alignment. One of:
%     r:		right
%     R:	right, floating
%     i:		inner border
%     I:		inner border, floating
%     l:		left
%     L:		left, floating
%     o:		outer border (default)
%     O:	outer border, floating
%
% 5. Optional parameter: Size as percentage
%     of line width. Over .8 (80%), no wrapfig will be
%     used.
%
%-----------------------------------------------------
% Sample:
%-----------------------------------------------------
%
% \begin{exhibit}[CVP, Price Adjusted]{Price Adjusted CVP}[][.41]
% Nulla malesuada porttitor diam. Donec felis erat, congue non, volutpat at,
% tincidunt tristique, libero. Vivamus viverra fermentum felis. Donec nonummy
% pellentesque ante.  \bigskip
% 
% \centerline{\colorbox{white}{\framebox{\includegraphics[width=0.9 \linewidth]{fig/cvp_variable_pricing.pdf}}}}
% \captionof{figure}[LabelInTOC]{FigureCaption}
% \captionsetup{font={footnotesize,it}}
% \vspace{-0.3cm}
% \captionof*{figure}{Source: Bla.}
% \label{fig:Label}
% \bigskip
% 
% Donec felis erat, congue non, volutpat at, tincidunt tristique, 
% \end{exhibit}
%
%-----------------------------------------------------
\ifpdf
\definecolor{exhibittitlebackground}{rgb}{.698,.780,.7411}
\definecolor{exhibittitlefont}{rgb}{1.,1,.7}
\definecolor{exhibitbodybackground}{rgb}{.733,.769,.847}
\else
%\Configure{HColor}{exhibittitlebackground}{rgb(.698,.780,.7411)}
%\Configure{HColor}{exhibittitlefont}{rgb(1.,1,.7)}
%\Configure{HColor}{exhibitbodybackground}{rgb(.733,.769,.847)}
\fi

\setlength{\intextsep}{0cm plus1cm minus1cm}

\newcommand{\listexhibitname}{List of Exhibits}
\newlistof{exhibit}{loe}{\listexhibitname}
\ifthenelse{\equal{\doctype}{book}}{
  \renewcommand{\theexhibit}{\arabic{chapter}.\arabic{exhibit}}
}{
  \renewcommand{\theexhibit}{\arabic{exhibit}}
}
\makeatletter
\NewDocumentEnvironment{exhibit}{O{} m O{} O{.5} O{o}}{%
\refstepcounter{exhibit}%
%
% If first parameter is empty, utilize caption for list of exhibits
%
\ifthenelse{\isempty{#1}{}}{\def\m@tcapt{#2}}{\def\m@tcapt{#1}}%
%
% Create variable for caption
%
\def\m@capt{#2}%
%
%
% Test for default value of alignment
%
\ifthenelse{\isempty{#5}}{\def\m@align{o}}{\def\m@align{#5}}%
% 
% Test for width
%
\ifthenelse{\isempty{#4}}{\def\m@width{.5}}{%
  \ifthenelse{\lengthtest{#4 pt  > .973 pt }}{%
    \def\m@width{.973}%
  }{%
    \def\m@width{#4}%
  }%
}%
%
% Add entry to list of exhibits
%
\ifpdf
\addcontentsline{loe}{exhibit}{\protect\numberline{\theexhibit}\m@tcapt}\par%
\fi
%
% Add wrapfig environment, but only if not too wide
%
\wcounte%
\ifpdf%
\ifthenelse{\lengthtest{\m@width pt  < .81 pt }}{%
  \ifthenelse{\isempty{#3}}{%
    \wrapfigure{\m@align}[0pt]{0pt}% Alignment: rRlLiIoO
  }{%
    \wrapfigure[#3]{\m@align}[0pt]{0pt}% Alignment: rRlLiIoO
  }%
}{%
  \center%
}%
\else%
  \center%
\fi%
\tabular{p{\m@width\textwidth}}\toprule%
\ifpdf%
\rowcolor{exhibittitlebackground}\color{exhibittitlefont}\raisebox{5pt}{%
\fi%
\fontsize{11}{13}\selectfont\textsc{Exhibit~\theexhibit: \m@capt}
\ifpdf%
}%
\fi%
\\\midrule %
\ifpdf%
\rowcolor{exhibitbodybackground}%
\fi%
\fontsize{10}{13}\selectfont%
}{%
\\\bottomrule\\\endtabular%
\wcounta%
%
% Finish wrapfig environment only if not too wide
%
\ifpdf%
\ifthenelse{\lengthtest{\m@width pt < .81 pt}}{%
  \endwrapfigure%
}{%
  \endcenter%
}%
\else%
  \endcenter%
\fi%
\vspace*{-\parskip}%
}%
\makeatother

