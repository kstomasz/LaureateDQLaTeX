%=====================================================
% Chapter 1: Introduction
% 
% This is not a re-hash of the question but an introduction to your submission
%
% 5% of the word count (50 words)
%=====================================================
% Chapter 1:  5 %  Introduction
% Chapter 2: 30 % Literature Review
% Chapter 3: 40 % Application of the Literature to the Question
% Chapter 4: 20 % Practical Experience
% Chapter 5: 10 % Conclusions
%=====================================================

\section{Introduction}\label{sec:Introduction}
% <content>%
\citeauthor{Vieira:2010ve}

To describe the balancing act of \gls{glos:liquidity} and \gls{glos:profitability} 
as a trade-off between short-term survival and realization of the ultimate goal
of a firm is too simplistic. As a result, section \vref{sec:LiteratureReview}
will add a number of influencing concepts such as risk-return theory,
term, size of company, industry and environmental factors. Section
\vref{sec:Application} will draw on these findings to improve the quality
of our answer. Section \vref{sec:PracticalExperience} will verify the
remaining hypotheses with regards to our own experiences both from
a \gls{sme} and a large corporation. Section \vref{sec:Conclusions}
concludes with an attempt to answer the question.

% To answer the question we are going to identify the different
% \glspl{glos:stakeholder} of accounting information, recent local and
% international scandals as well as counter measures (section
% \vref{sec:LiteratureReview}).
% We are then going to apply this information to improve the quality of
% our answer (section \vref{sec:Application}).
% Section \vref{sec:PracticalExperience} combines these findings with our own
% experiences both from a \gls{sme} and from a large corporation. Section
% \vref{sec:Conclusions} concludes with an attempted answer to the question. 


%</content>%













