%=====================================================
% Chapter 3: Application of the Literature to the Question
% 
% You use the literature that you have found, together with your own views 
% to formulate your answer
%
% 40% of the word count (400 words)
%=====================================================

\section{Application} \label{sec:Application}
% <content>%

\subsection{Case Study} \label{sec:case_study}

\citeauthor{Vieira:2010ve} applied the findings to the airline industry in the context
of the Lehman crisis (2008) to verify three hypotheses (\citeyear[15]{Vieira:2010ve}):

\begin{enumerate}
  \item \emph{Hypothesis:} ``On the short term  the relationship between liquidity and profitability is negative.'' \label{h1}
  \item \emph{Hypothesis:} ``On the medium term a low liquidity level will derail the upkeep of high 
  profitability, and also a low profitability will derail the upkeep of a high liquidity.'' \label{h2}
  \item \emph{Hypothesis:}``Over the year of 2008, the companies with higher liquidity would be able to
  achieve a better performance.''  \label{h3}
\end{enumerate}

\begin{figure}[htp]
\centerline{\framebox{\includegraphics[width=9cm]{fig/plquadrant.jpg}}}
\caption[Liquidity--Profitability--Matrix]{Liquidity--Profitability--Matrix}
\captionsetup{font={footnotesize,it}}
\vspace{-0.3cm}
\caption*{Source: \cite[19]{Vieira:2010ve}}
\label{fig:plquadrant}
\end{figure}

\ni To cluster the companies he observed, \citeauthor{Vieira:2010ve} created a Liquidity--Profitability--Matrix (see
figure \vref{fig:plquadrant}).\endnote{\citeauthor{Vieira:2010ve} clustered the companies using the following rules:

\begin{itemize}
  \item Companies with liquidity ratio $> 1$ go in row \emph{H}, the others in row \emph{L}
  \item Companies with an average \gls{roa} for the observation period higher than the average
  market \gls{roa} went in row \emph{h}, the others in row \emph{l}.
\end{itemize}

\citeauthor{Vieira:2010ve} hypothesized the following behaviors (\citeyear[19]{Vieira:2010ve}):

\begin{itemize}
  \item Companies from quadrant 2 (Hl) would migrate to the quadrants 3 (lH) or 4 (Ll); and companies
  from quadrant 3 (Lh) would migrate to quadrants 2 (Hl) or 4 (Ll) \emph{because the low level of
  one of the indicators would deteriorate the other}.
  \item Companies from quadrants 1 (Hh) or 4 (Ll) would stay there \emph{as their extreme position---either
  good or bad---would lock them in and make it hard to change.}  
\end{itemize}}


\subsection{Results} \label{sec:results}

\subsubsection{Hypothesis 1: Short Term Negative Relationship between Liquidity and Profitability} \label{sec:res_h1}

Interestingly,  hypothesis  \vref{h1} was \emph{clearly rejected}: ``In fact it was
found a significant positive relationship between the indicators.''
(\citeyear[24]{Vieira:2010ve}) This result goes against the literature (see
sections \vref{sec:liquidity_profitability_trade_off} and
\vref{sec:risk_and_returns}), and \citeauthor{Vieira:2010ve} hypothesizes that
the studied industry sector (airlines) might differ significantly from other
sector.\endnote{As an example, this may be due to a high demand on current expenses (fuel, maintenance), hence a
high level of \gls{glos:workingcapital} (see section \vref{sec:cash_gap}) would
be directly related to reducing costs and obtaining higher profits; also, these
companies being rather large, they would be less likely to show a high demand on
liquidity compared to \glspl{glos:sme}---see also \citeauthor{Decman:2012vn} who
note that ``\Glspl{sme} are usually characterized by a high proportion of
current assets to total assets, less liquid, exposed to high volatility of cash
flows and mostly rely on short-term borrowing.'' (\citeyear[692]{Decman:2012vn}).}


\subsubsection{Hypothesis 2: Medium Term, low Liquidity will deteriorate Profitability, and vice versa} \label{sec:res_h2}

Hypothesis \vref{h2} was \emph{confirmed}, yet due to the rejection of hypothesis
\vref{h1}, ``lost part of its meaning:`` \citep[32]{Vieira:2010ve}: An inversion
was not observed, yet a dependency of the medium term relationship on the short
term results.

\subsubsection{Hypothesis 3: Over the year 2008, a higher liquidity helped achieve better performance} \label{sec:res_h3}



The performance indicators observed \emph{strongly confirmed} the hypothesis, as
companies with a higher liquidity had a comparatively better performance in
2008, reinforcing the view that ``the importance of the management of the
working capital increases during hard times.''
(\citeyear[31]{Vieira:2010ve})\endnote{Yet, as hypothesis \vref{h1} was
rejected, this ``lost its original sense, since it was expected that while the
relationship would be negative on the years of prosperity and then become
positive on the years of economic decline.'' (\citeyear[30]{Vieira:2010ve}) }



%</content>%


















