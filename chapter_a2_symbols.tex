%=====================================================
% Symbols
%=====================================================

%-----------------------------------------------------
% Samples
%-----------------------------------------------------

% Usage:
% \section{Some Greek symbols}
% If you calculate with \gls{symb:Pi} you always get an irrational result, because 
% \gls{symb:Pi} itself is irrational. As a matter of fact, there are \gls{symb:Phi} 
% and \gls{symb:Lambda}, too.

%Some entries for the list of symbols
\newglossaryentry{symb:Pi}{
  type=symbolslist,
  name=$\pi$,
  description={A mathematical constant whose value is the ratio of any circle's circumference to its diameter.},
  sort=symbolpi
}
\newglossaryentry{symb:Phi}{
  type=symbolslist,
  name=$\varphi$,
  description={An angle.},
  sort=symbolphi
}
\newglossaryentry{symb:Lambda}{
  type=symbolslist,
  name=$\lambda$,
  description={Lambda indictes an eigenvalue in the mathematics  of linear algebra.},
  sort=symbollambda
}

%-----------------------------------------------------
% Content
%-----------------------------------------------------

%<content>%



%</content>%


%-----------------------------------------------------
% Default Content
%-----------------------------------------------------


%-----------------------------------------------------
% AP
%-----------------------------------------------------

\def\glosdescap{Also called \emph{Actual Rate}. This is a 
calculated value that describes the price, or rate, of a given
cost factor, such as nursing, on an per-hour basis. As an \emph{actual}
value, it is calculated as follows:
 
\begin{equation}
   AP = \frac{\text{Actual Costs}}{\text{Total Actual Hours}}
\end{equation} 
}
\newglossaryentry{glos:ap}{
  name=Actual Price,
  description={\glosdescap}
}
\def\symbdescap{See \glslink{glos:ap}{Actual Price}.}
\newglossaryentry{symb:ap}{
  type=symbolslist,
  name=AP,
  description={\symbdescap},
  sort=symbolap
}

%-----------------------------------------------------
% AQ
%-----------------------------------------------------
\def\glosdescaq{This is a 
measured value that describes the \emph{actual} quantity
of a product in monetary terms; for observations of working
hours, it is typically called  \emph{Actual Time}.
}
\newglossaryentry{glos:aq}{
  name=Actual Quantity,
  description={\glosdescaq}
}
\def\symbdescaq{See \glslink{glos:aq}{Actual Quantity}.}
\newglossaryentry{symb:aq}{
  type=symbolslist,
  name=AQ,
  description={\symbdescaq},
  sort=symbolaq
}


%-----------------------------------------------------
% SP
%-----------------------------------------------------

\def\glosdescsp{Also called \emph{Standard Rate}. This is a 
calculated value that describes the price, or rate, of a given
cost factor, such as nursing, on an per-hour basis. As a \emph{standard},
or \emph{budgeted} value, it is calculated as follows:
 
\begin{equation}
   SP = \frac{\text{Budgeted Costs}}{\text{Total Budgeted Hours}}
\end{equation} 
}
\newglossaryentry{glos:sp}{
  name=Standard Price,
  description={\glosdescsp}
}
\def\symbdescsp{See \glslink{glos:sp}{Standard Price}.}
\newglossaryentry{symb:sp}{
  type=symbolslist,
  name=SP,
  description={\symbdescsp},
  sort=symbolsp
}



%-----------------------------------------------------
% SQ
%-----------------------------------------------------
\def\glosdescsq{This is a 
measured value that describes the \emph{budgeted} quantity
of a product in monetary terms; for observations of working
hours, it is typically called  \emph{Standard Time}.
If the \glslink{glos:ao}{actual output} differs
significantly from the \glslink{glos:so}{budgeted output},
it is necessary to utilize \glslink{glos:rsq}{Revised Standard Quantity}
instead.
}
\newglossaryentry{glos:sq}{
  name=Standard Quantity,
  description={\glosdescsq}
}
\def\symbdescsq{See \glslink{glos:sq}{Standard Quantity}.}
\newglossaryentry{symb:sq}{
  type=symbolslist,
  name=SQ,
  description={\symbdescsq},
  sort=symbolsq
}

%-----------------------------------------------------
% RSQ
%-----------------------------------------------------
\def\glosdescrsq{This is a calculated value that
describes the \emph{budgeted} quantity of a  
product in monetary terms, corrected by the
ratio of \glslink{glos:ao}{actual output} to
\glslink{glos:so}{budgeted output}. For observations
of working hours, it is typically called \emph{Revised Standard Time}.
It is calculated as follows:
 
\begin{equation}
  RSQ = \glslink{symb:sq}{SQ} \: \frac{\glslink{symb:ao}{AO}}{\glslink{symb:so}{SO}}
\end{equation} 

Since the equation resolves to the value of \glslink{symb:sq}{SQ} if
\glslink{symb:ao}{AO} and \glslink{symb:so}{SO} are identical,
RSQ  should always be used instead of \glslink{symb:sq}{SQ}.
}
\newglossaryentry{glos:rsq}{
  name=Revised Standard Quantity,
  description={\glosdescrsq}
}
\def\symbdescrsq{See \glslink{glos:rsq}{Revised Standard Quantity}.}
\newglossaryentry{symb:rsq}{
  type=symbolslist,
  name=RSQ,
  description={\symbdescrsq},
  sort=symbolrsq
}

%-----------------------------------------------------
% SO
%-----------------------------------------------------
\def\glosdescso{Also called \emph{Standard Yield}. This is a 
measured value that describes the \emph{budgeted} output,
in terms of quantity, of a given activity. 
}
\newglossaryentry{glos:so}{
  name=Standard Output,
  description={\glosdescso}
}
\def\symbdescso{See \glslink{glos:so}{Standard Output}.}
\newglossaryentry{symb:so}{
  type=symbolslist,
  name=SO,
  description={\symbdescso},
  sort=symbolso
}

%-----------------------------------------------------
% AO
%-----------------------------------------------------
\def\glosdescao{Also called \emph{Actual Yield}. This is a 
measured value that describes the \emph{actual} output,
in terms of quantity, of a given activity.  
}
\newglossaryentry{glos:ao}{
  name=Actual Output,
  description={\glosdescao}
}
\def\symbdescao{See \glslink{glos:ao}{Actual Output}.}
\newglossaryentry{symb:ao}{
  type=symbolslist,
  name=AO,
  description={\symbdescao},
  sort=symbolao
}

%-----------------------------------------------------
% TPU
%-----------------------------------------------------
\def\glosdesctpu{Also called \emph{Productivity}. This is a 
\emph{measured} value that describes the time required to
produce one unit of output.
}
\newglossaryentry{glos:tpu}{
  name=Time per Unit,
  description={\glosdesctpu}
}
\def\symbdesctpu{See \glslink{glos:tpu}{Time per Unit}.}
\newglossaryentry{symb:tpu}{
  type=symbolslist,
  name=TPU,
  description={\symbdesctpu},
  sort=symboltpu
}

%-----------------------------------------------------
% OCV
%-----------------------------------------------------
\def\glosdescocv{This is a 
calculated value describes the difference between
\emph{actual}  expenses and \emph{budgeted}
expenses in monetary terms.

It is calculated as follows:\footnote{The calculation was
derived from \cite{Averkamp:2012locv}.}
 
\begin{equation}
   OCV = \glslink{symb:rsq}{RSQ} \times \glslink{symb:sp}{SP} -
                              \glslink{symb:aq}{AQ} \times \glslink{symb:sp}{AP}
\end{equation} 

\begin{dangerous}
A negative value is called \emph{unfavorable}. It is considered to be in the responsibility of
\emph{departmental management}  as far as they are able to exert control over the costs contributing to
the variance.
\end{dangerous}
}
\newglossaryentry{glos:ocv}{
  name=Overhead Controllable Variance,
  description={\glosdescocv}
}
\def\symbdescocv{See \glslink{glos:ocv}{Overhead Controllable Variance}.}
\newglossaryentry{symb:ocv}{
  type=symbolslist,
  name=OCV,
  description={\symbdescocv},
  sort=symbolocv
}

%-----------------------------------------------------
% OVV
%-----------------------------------------------------
\def\glosdescovv{This is a 
calculated value that describes the difference between the \emph{budget}  allowed
and the \emph{budgeted}  expenses to the \emph{actual} work in progress.

It is calculated as follows:\footnote{The calculation was
derived from \cite{Averkamp:2012ovv}.}
 
\begin{equation}
   OVV = \glslink{symb:sp}{SP}_{fixed\:costs} \times \left( \glslink{symb:ao}{AO} \times
   \glslink{symb:tpu}{Productivity} - \glslink{symb:rsq}{RSQ}   \right)
\end{equation}

\begin{dangerous}
A negative value is called \emph{unfavorable}. In that case, the value expresses the cost of 
capacity available but not utilized efficiently. It is considered to be in the responsibility of
\emph{executive management}  and \emph{departmental management}.
\end{dangerous}
}
\newglossaryentry{glos:ovv}{
  name=Overhead Volume Variance,
  description={\glosdescovv}
}
\def\symbdescovv{See \glslink{glos:ovv}{Overhead Volume Variance}.}
\newglossaryentry{symb:ovv}{
  type=symbolslist,
  name=OVV,
  description={\symbdescovv},
  sort=symbolovv
}

%-----------------------------------------------------
% OSV
%-----------------------------------------------------
\def\glosdescosv{This is a 
calculated value that describes the difference between the \emph{actual}
expenses expenses and the \emph{budget} allowed based on the \emph{actual}
quantity produced or hours worked, respectively.   

It is calculated as follows:\footnote{The calculation was
derived from \cite{Averkamp:2012osv}.}
 
\begin{equation}
  OSV = \glslink{symb:aq}{AQ} \times \glslink{symb:ap}{AP} -
              \glslink{symb:aq}{AQ} \times \glslink{symb:ap}{AP}_{overhead} -
              \text{Actual Variable Costs}
\end{equation}

\begin{dangerous}
A negative value is called \emph{unfavorable}. It is considered to be in the area of
responsibility of the \emph{department manager} who has to keep actual expenses
within budgeted limits.
\end{dangerous}
}
\newglossaryentry{glos:osv}{
  name=Overhead Spending Variance,
  description={\glosdescosv}
}
\def\symbdescosv{See \glslink{glos:osv}{Overhead Spending Variance}.}
\newglossaryentry{symb:osv}{
  type=symbolslist,
  name=OSV,
  description={\symbdescosv},
  sort=symbolosv
}


%-----------------------------------------------------
% OEV
%-----------------------------------------------------
\def\glosdescoev{This is a 
calculated value that describes the difference between the \emph{actual}
expenses expenses and the \emph{budget} allowed based on the \emph{actual}
quantity produced or hours worked, respectively.   

It is calculated as follows:\footnote{The calculation was
derived from \cite{Averkamp:2012oev}.}
 
\begin{equation}
  OEV = \left( \glslink{symb:ap}{AP}_{overhead} + \glslink{symb:sp}{SP}_{fixed\:costs} \right) \times
              \left(  \glslink{symb:aq}{AQ} - \glslink{symb:ao}{AO} \times
              \glslink{symb:tpu}{Productivity}  \right)
\end{equation}

\begin{dangerous}
A negative value is called \emph{unfavorable} and typically caused by inefficiencies
such as inexperienced labor, changes in operations, introduction of new procedures, 
work tools or materials. It is considered to be in the area of
responsibility of the \emph{department manager}.
\end{dangerous}
}
\newglossaryentry{glos:oev}{
  name=Overhead Efficiency Variance,
  description={\glosdescoev}
}
\def\symbdescoev{See \glslink{glos:oev}{Overhead Efficiency Variance}.}
\newglossaryentry{symb:oev}{
  type=symbolslist,
  name=OEV,
  description={\symbdescoev},
  sort=symboloev
}

%-----------------------------------------------------
% OIV
%-----------------------------------------------------
\def\glosdescoiv{This is a 
calculated value that describes the difference between the
budget allowed based on \emph{actual} hours worked, multiplied
with the \emph{budgeted} overhead rate. 

As an \emph{actual}
value, it is calculated as follows:\footnote{The calculation was
derived from \cite{Averkamp:2012oiv}.}
 
\begin{equation}
  OIV = \left( \glslink{symb:aq}{AQ} - \glslink{symb:rsq}{RSQ} \right) \times
             \glslink{symb:sp}{SP}_{fixed\:costs}
\end{equation}

\begin{dangerous}
A negative value is called \emph{unfavorable} and indicates the 
amount of overhead that is under absorbed due to \emph{actual}
hours being lower than \emph{budgeted} hours on which the
calculation of the overhead rate was based. It is considered to be in the area of
responsibility of the \emph{department manager}.
\end{dangerous}
}
\newglossaryentry{glos:oiv}{
  name=Overhead Idle Capacity Variance,
  description={\glosdescoiv}
}
\def\symbdescoiv{See \glslink{glos:oiv}{Overhead Idle Capacity Variance}.}
\newglossaryentry{symb:oiv}{
  type=symbolslist,
  name=OIV,
  description={\symbdescoiv},
  sort=symboloiv
}


%-----------------------------------------------------
% OOV
%-----------------------------------------------------
\def\glosdescoov{Also called \emph{Net Factory Overhead Variance}. 
An overall figure, it is considered to be in the responsibility of
\emph{executive management} and is a 
calculated value that describes the difference between the \emph{actual} 
overhead and expenses incurred using the \emph{budgeted} overhead rate. 

It is calculated as follows:\footnote{The calculation was
derived from \cite{Averkamp:2012oov}.}
 
\begin{equation}
  OOV = \left( \glslink{symb:ap}{AP}_{overhead} + \glslink{symb:sp}{SP}_{fixed\:costs} \right) \times
              \glslink{symb:ao}{AO} \times \glslink{symb:tpu}{Productivity} -
              \glslink{symb:aq}{AQ} \times \glslink{symb:ap}{AP}
\end{equation}

\begin{dangerous}
A negative value is called \emph{unfavorable} and typically requires further
analysis to guide management to solve the situation. In practice, this requires
to look at the other overhead variances, such as \glslink{symb:ocv}{OCV},
 \glslink{symb:ovv}{OVV} (two-variances method), but also  \glslink{symb:osv}{OSV},
 \glslink{symb:oev}{OEV} and \glslink{symb:oiv}{OIV} (three-variances method)
 to understand better the causes of the issues; using the \glslink{symb:osv}{OSV}
 and  variable \glslink{symb:oev}{OEV} on one side and the fixed  \glslink{symb:oev}{OEV}
 and   \glslink{symb:oiv}{OIV} (four-variances method), one splits the 
  \glslink{symb:oev}{OEV}  into its fixed and variable components.
\end{dangerous}
}
\newglossaryentry{glos:oov}{
  name=Overall Overhead Variance,
  description={\glosdescoov}
}
\def\symbdescoov{See \glslink{glos:oov}{Overall Overhead Variance}.}
\newglossaryentry{symb:oov}{
  type=symbolslist,
  name=OOV,
  description={\symbdescoov},
  sort=symboloov
}

%-----------------------------------------------------
% LCV
%-----------------------------------------------------
\def\glosdesclcv{This  is a 
calculated value that describes the difference between the \emph{budgeted} cost
for production and the \emph{actual} cost for production.  

Comparing Labor costs with Material costs,
LCV corresponds to \glslink{symb:mcv}{MCV}.

It is calculated as follows:\footnote{The calculation was
derived from \cite{Poudel:2012lcv}.}
 
\begin{equation}
  LCV = \left( \glslink{symb:rsq}{RSQ} \times \glslink{symb:sp}{SP}\right) - 
              \left( \glslink{symb:aq}{AQ} \times \glslink{symb:ap}{AP}\right)
\end{equation}

\begin{dangerous}
A negative value is called \emph{unfavorable} and typically means that
more hours are utilized than are allowed budgeted for, pointing to
possibly inefficient use of labor time due to lack of automation
or otherwise inefficient production methods (\glslink{glos:lev}{LEV})
and possibly increased hourly charges \emph{actual} vs. \emph{budgeted}
(\glslink{glos:lrv}{LRV}).   
\end{dangerous}
}
\newglossaryentry{glos:lcv}{
  name=Labor Cost Variance,
  description={\glosdesclcv}
}
\def\symbdesclcv{See \glslink{glos:lcv}{Labor Cost Variance}.}
\newglossaryentry{symb:lcv}{
  type=symbolslist,
  name=LCV,
  description={\symbdesclcv},
  sort=symbollcv
}

%-----------------------------------------------------
% MCV
%-----------------------------------------------------
\def\glosdescmcv{This  is a 
calculated value that describes the difference between the \emph{budgeted} cost
for production and the \emph{actual} cost for production.  

Comparing Labor costs with Material costs,
MCV corresponds to \glslink{symb:lcv}{LCV}.

It is calculated as follows:\footnote{The calculation was
derived from \cite{Poudel:2011mcv}.}
 
\begin{equation}
  MCV = \left( \glslink{symb:rsq}{RSQ} \times \glslink{symb:sp}{SP}\right) - 
              \left( \glslink{symb:aq}{AQ} \times \glslink{symb:ap}{AP}\right)
\end{equation}

\begin{dangerous}
A negative value is called \emph{unfavorable} and typically means that
more material is utilized than are allowed budgeted for, pointing to
possibly inefficient use of material due to inefficient production 
methods (\glslink{glos:muv}{MUV})
and possibly increased material costs \emph{actual} vs. \emph{budgeted}
(\glslink{glos:mpv}{MPV}).   
\end{dangerous}
}
\newglossaryentry{glos:mcv}{
  name=Material Cost Variance,
  description={\glosdescmcv}
}
\def\symbdescmcv{See \glslink{glos:mcv}{Material Cost Variance}.}
\newglossaryentry{symb:mcv}{
  type=symbolslist,
  name=MCV,
  description={\symbdescmcv},
  sort=symbolmcv
}

%-----------------------------------------------------
% LRV
%-----------------------------------------------------
\def\glosdesclrv{This  is a 
calculated value that describes the difference between the \emph{budgeted} cost
and the \emph{actual} cost paid for the \emph{actual} number of hours.

Comparing Labor costs with Material costs,
LRV corresponds to \glslink{symb:mpv}{MPV}.

It is calculated as follows:\footnote{The calculation was
derived from \cite{Poudel:2012lrv}.}
 
\begin{equation}
  LRV =  \glslink{symb:aq}{AQ} \times 
              \left( \glslink{symb:sp}{SP} - \glslink{symb:ap}{AP}\right)
\end{equation}

\begin{dangerous}
A negative value is called \emph{unfavorable} and typically points to
increased \emph{actual} hourly rates compared to what was
\emph{budgeted}.
\end{dangerous}
}
\newglossaryentry{glos:lrv}{
  name=Labor Rate Variance,
  description={\glosdesclrv}
}
\def\symbdesclrv{See \glslink{glos:lrv}{Labor Rate Variance}.}
\newglossaryentry{symb:lrv}{
  type=symbolslist,
  name=LRV,
  description={\symbdesclrv},
  sort=symbollrv
}

%-----------------------------------------------------
% MPV
%-----------------------------------------------------
\def\glosdescmpv{This  is a 
calculated value that describes the difference between the \emph{budgeted} cost
and the \emph{actual} cost paid for the \emph{actual} material utilized.

Comparing Labor costs with Material costs,
MPV corresponds to \glslink{symb:lrv}{LRV}.

It is calculated as follows:\footnote{The calculation was
derived from \cite{Poudel:2011mpv}.}
 
\begin{equation}
  MPV =  \glslink{symb:aq}{AQ} \times 
              \left( \glslink{symb:sp}{SP} - \glslink{symb:ap}{AP}\right)
\end{equation}

\begin{dangerous}
A negative value is called \emph{unfavorable} and typically points to
increased \emph{actual} price for material compared to what was
\emph{budgeted}.
\end{dangerous}
}
\newglossaryentry{glos:mpv}{
  name=Material Price Variance,
  description={\glosdescmpv}
}
\def\symbdescmpv{See \glslink{glos:mpv}{Material Price Variance}.}
\newglossaryentry{symb:mpv}{
  type=symbolslist,
  name=MPV,
  description={\symbdescmpv},
  sort=symbolmpv
}

%-----------------------------------------------------
% LEV
%-----------------------------------------------------
\def\glosdesclev{This  is a 
calculated value that compares the \emph{actual} number of hours it took to
create an actual output with the number of hours \emph{budgeted}
for that output.      

Comparing Labor costs with Material costs,
LEV corresponds to \glslink{symb:muv}{MUV}.

It is calculated as follows:\footnote{The calculation was
derived from \cite{Poudel:2011lev}.}
 
\begin{equation}
  LEV =  \glslink{symb:aq}{AQ} \times 
              \left( \glslink{symb:sp}{SP} - \glslink{symb:ap}{AP}\right)
\end{equation}

\begin{dangerous}
A negative value is called \emph{unfavorable} and typically means that
more hours are utilized than are allowed budgeted for, pointing to
possibly inefficient use of labor time due to lack of automation
or otherwise inefficient production methods.
\end{dangerous}
}
\newglossaryentry{glos:lev}{
  name=Labor Efficiency Variance,
  description={\glosdesclev}
}
\def\symbdesclev{See \glslink{glos:lev}{Labor Efficiency Variance}.}
\newglossaryentry{symb:lev}{
  type=symbolslist,
  name=LEV,
  description={\symbdesclev},
  sort=symbollev
}

%-----------------------------------------------------
% MUV
%-----------------------------------------------------
\def\glosdescmuv{This  is a 
calculated value that compares the \emph{actual} amount of material it took to
create an actual output with the amoutn of material \emph{budgeted}
for that output.      

Comparing Labor costs with Material costs,
MUV corresponds to \glslink{symb:lev}{LEV}.

It is calculated as follows:\footnote{The calculation was
derived from \cite{Poudel:2011muv}.}
 
\begin{equation}
  MUV =  \glslink{symb:aq}{AQ} \times 
              \left( \glslink{symb:sp}{SP} - \glslink{symb:ap}{AP}\right)
\end{equation}

\begin{dangerous}
A negative value is called \emph{unfavorable} and typically means that
more material is utilized than are allowed budgeted for, pointing to
possibly inefficient use of material due to inefficient production methods.
\end{dangerous}
}
\newglossaryentry{glos:muv}{
  name=Material Usage Variance,
  description={\glosdescmuv}
}
\def\symbdescmuv{See \glslink{glos:muv}{Material Usage Variance}.}
\newglossaryentry{symb:muv}{
  type=symbolslist,
  name=MUV,
  description={\symbdescmuv},
  sort=symbolmuv
}

%-----------------------------------------------------
% LYV
%-----------------------------------------------------
\def\glosdesclyv{This  is a 
calculated value that identifies the portion of the
\glslink{glos:lev}{LEV} to obtain a favorable
or unfavorable yield.

Comparing Labor costs with Material costs,
LYV corresponds to \glslink{symb:myv}{MYV}.

It is calculated as follows:\footnote{The calculation was
derived from \cite{Poudel:2011lyv}.}
 
\begin{equation}
  LYV =  \glslink{symb:sp}{SP} \times 
              \left( \glslink{symb:ao}{AO} - \glslink{symb:so}{SO}\right)
\end{equation}

\begin{dangerous}
A negative value is called \emph{unfavorable} and typically 
points to inefficiencies---see \glslink{glos:lev}{LEV}---when
creating a given output.
\end{dangerous}
}
\newglossaryentry{glos:lyv}{
  name=Labor Yield Variance,
  description={\glosdesclyv}
}
\def\symbdesclyv{See \glslink{glos:lyv}{Labor Yield Variance}.}
\newglossaryentry{symb:lyv}{
  type=symbolslist,
  name=LYV,
  description={\symbdesclyv},
  sort=symbollyv
}

%-----------------------------------------------------
% MYV
%-----------------------------------------------------
\def\glosdescmyv{This  is a 
calculated value that identifies the portion of the
\glslink{glos:muv}{MUV} to obtain a favorable
or unfavorable yield.

Comparing Labor costs with Material costs,
MYV corresponds to \glslink{symb:lyv}{LYV}.

It is calculated as follows:\footnote{The calculation was
derived from \cite{Poudel:2011myv}.}
 
\begin{equation}
  MYV =  \glslink{symb:sp}{SP} \times 
              \left( \glslink{symb:ao}{AO} - \glslink{symb:so}{SO}\right)
\end{equation}

\begin{dangerous}
A negative value is called \emph{unfavorable} and typically 
points to inefficiencies---see \glslink{glos:muv}{MUV}---when
creating a given output.
\end{dangerous}
}
\newglossaryentry{glos:myv}{
  name=Material Yield Variance,
  description={\glosdescmyv}
}
\def\symbdescmyv{See \glslink{glos:myv}{Material Yield Variance}.}
\newglossaryentry{symb:myv}{
  type=symbolslist,
  name=MYV,
  description={\symbdescmyv},
  sort=symbolmyv
}










































