%=====================================================
% Chapter 2: Literature Review
% 
% Summary of the relevant articles you have located in the library that can 
% be used to help you address the question.
%
% 30% of the word count (300 words)
%=====================================================

\section{Literature Review} \label{sec:LiteratureReview}
% <content>%

\subsection{Liquidity--Profitability Trade--Off} \label{sec:liquidity_profitability_trade_off}

According to \citeauthor{Braga:1995rr}, ``the balance between adequate
\gls{glos:liquidity} and satisfactory \gls{glos:profitability} is a constant
challenge to financial management.'' (\citeyear[31]{Braga:1995rr})
\citeauthor{Arnold:2008wd}  (\citeyear[537]{Arnold:2008wd}) gives three
fundamental reasons for holding cash:

\begin{itemize}
  \item To pay \emph{daily expenses}  such as salaries, materials, and taxes.
  \item To cope with \emph{uncertainty}  and eventual downturns with regards
  to future cash flows.
  \item To allow for \emph{highly profitable investments}  what would require
  \emph{immediate payment}.
\end{itemize}

If the company fails to maintain a sufficiently high level of
\gls{glos:workingcapital}, i.e., the amount by which the firms current assets
exceed its current liabilities, the company becomes vulnerable to short term
financial shortages and is hence likely to become insolvent; in other words, the
company must provide for a reasonable margin of safety so that it can cover its
liabilities. In very simple terms, the foremost concern of the executive will be
to make sure that enough \emph{current} funds are available to cope with the
day-to-day contingencies.\endnote{While a high liquidity is normally regarded as
a sign for the financial strength of a company, if the value is too high it can
be understood as a symptom for the current assets of a company being less
profitable than the fixed assets, which means that money invested in current
assets generates less returns than fixed assets, hence representing an
opportunity cost.}


\subsection{Cash Gap and Industry Variation} \label{sec:cash_gap}

More importantly, \citeauthor{Losbichler:2012zr} explains the trade-off between
liquidity and profitability when discussing ``the average time required to turn
a dollar invested in raw material into a dollar collected from customers''
(\citeyear[28]{Losbichler:2012zr})---see figure \vref{fig:c2c}: If the C2C cycle
time is short, the required working capital is low, hence the company is
effective at managing the working capital
required.\endnote{\citeauthor{Losbichler:2012zr} explain how ``effective
management of working capital is an important driver of the company's
profitability. Lower working capital would allow companies to operate at lower
profit margins while earning the same or higher profitability overall,''
(\citeyear[30]{Losbichler:2012zr}) and generally, ``an actively managed cash gap
is essential to profitable growth.'' \citep[32]{Boer:1999ly}}

\begin{figure}[htp]
\centerline{\framebox{\includegraphics[width=15.7cm]{fig/c2c.jpg}}}
\caption[Cash-to-Cash Cycle Time Metric]{The Cash-to-Cash (C2C) Cycle Time Metric}
\captionsetup{font={footnotesize,it}}
\vspace{-0.3cm}
\caption*{Source: \cite[29]{Losbichler:2012zr}}
\label{fig:c2c}
\end{figure}


Likewise, \citeauthor{Eljelly:2004fk} has studied the correlation of
profitability and liquidity utilizing a regression analysis on
\gls{glos:currentratio} and cash conversion cycle; they observed
a \emph{negative} relation between profitability and liquidity, and
also they saw that the cash conversion cycle had a bigger impact on
profitability than the \gls{glos:currentratio}. This was strongly industry
dependent.


\subsection{Risks and Returns and Term Dependency} \label{sec:risk_and_returns}

While high liquidity reduces risk of running out of cash, risk and profitability are
related (i.e., the higher the risk of an investment, the higher the profits that could
be collected). As a result, ``a firm is required to maintain a balance between
liquidity and profitability in its day-to-day operations.'' \citep[35]{Niresh:2012uq}

\citeauthor{Pimentel:2005pd} studied the trade-off between profitability and liquidity
on a medium- and long-term basis, assuming due to the interaction between the two,
low liquidity will adversely affect profitability, and vice versa.






% ``The lender may require the business to maintain a certain level of liquidity
% during the period of the loan. This would typically be a requirement that the
% borrower business's current ratio is maintained at, or above, a specified level.'' \citep[330]{Atrill:2010ys}


%</content>%






















