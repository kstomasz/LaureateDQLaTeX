%=====================================================
% Title Page
%=====================================================

%=====================================================
% Configuration
%=====================================================


\newgeometry{bottom=0cm,left=0.5cm,right=0cm,top=5cm}
\def\mnbook{Final~Project~Paper}
\def\mnsubtitle{Accounting and Financial Analysis}
\def\mnsubsubtitle{Module Project}
\def\mnmoduleweek{Week 7}
\def\mncreated{June 4, 2013}
\def\mnversion{0.1}

\def\mnname{Matthias Nott, University of Liverpool}
\def\mnstudid{H00023837}
\def\mnmail{\href{mailto:m.nott@liverpool.ac.uk}{m.nott@liverpool.ac.uk}}
\def\mnauthor{Matthias Nott}
\def\mnstate{Under Development. Not ready for production.}

\pdfbookmark[0]{\mnbook}{toc}
\begin{center}
\vspace*{0cm}
\thispagestyle{empty}
\vspace*{-2cm}
{ \fontsize{39.876}{42}\selectfont \bf \mnbook\\}
{\sc \Large --- \mnsubtitle\ ---\\[0.2cm]}\vspace*{6cm}
\includegraphics[width=8.7cm]{fig/microplus-logo.png}\\[2.8cm]
 
\vspace*{4cm}
%{\large \mnsubsubtitle}\\
%{\it \mnmoduleweek}
\end{center}
%\vspace*{1cm}
\centerline{
\adjustbox{raise=-1cm}{
\begin{tabular}{lll}\\
  \multicolumn{3}{c}{\bf \mnauthor}\\\hline
  %&\\ 
  \rule{0pt}{3ex}Contact&: &  \mnmail \\
  Student ID&: & \mnstudid \\
  Created&: & \mncreated \\
  Updated&: & \today    \\
  Version&: & \mnversion \\
  Word Count&: & \input{tmp/wc.tex}Words\\
  \end{tabular}
}}
 
 
\clearpage{\pagestyle{empty}}
\restoregeometry
