%=====================================================
% Static Glossary etc. entries
%=====================================================
%
% Especially interesting if you have acronyms which are only introduced
% within the (later) glossary, hence would not have been used in the text,
% and hence would not appear in the list of acronyms. If you define them
% here, using \glslink{acronymname}{\null}, nothing will be output, yet
% the link will be established, hence the entry will be printed. Likewise,
% you can put here additional things you want to have in the list of acronyms
% (or glossary or symbols, for that matter) even if you do not reference them
% in the text (for whatever reason).
%
%=====================================================

%-----------------------------------------------------
% Samples
%-----------------------------------------------------

% Usage:
%
% \glslink{rosf}{\null}
% \glslink{roce}{\null}
% \glslink{cogs}{\null}
% \glslink{glos:costofgoodstosales}{\null}
% \glslink{glos:markupratio}{\null}
% \glslink{glos:operatingprofitmargin}{\null}
% \glslink{sga}{\null}
% \glslink{glos:netprofit}{\null}
% \glslink{bep}{\null}
% \glslink{glos:bep}{\null}
% \glslink{cvp}{\null}
% \glslink{glos:cvp}{\null}
% \glslink{sga}{\null}
% \glslink{arr}{\null}
% \glslink{pp}{\null}
% \glslink{npv}{\null}
% \glslink{irr}{\null}
% \glslink{wacc}{\null}
% \glslink{ro}{\null}

%-----------------------------------------------------
% Content
%-----------------------------------------------------

%<content>%



%</content>%



































